\documentclass[12pt]{turabian-researchpaper}
\usepackage{biblatex-chicago}
\renewcommand{\thesection}{\Roman{section}}
\renewcommand{\thesubsection}{\thesection.\Roman{subsection}}

\renewcommand{\bibsetup}{\singlespacing}
\renewcommand{\bibitemsep}{1\baselineskip}
\renewcommand{\bibhang}{0.5in}	

\addbibresource{sda.bib}


\begin{document}


\begin{singlespace}
\noindent Alden Peterson \newline
\noindent HT3300 - INTRO TO APOLOGETICS \newline
\noindent SDA 2 \newline
\noindent\rule{4cm}{0.4pt}
\end{singlespace}


\section{Guided Reflection}
% In 1-2 pages, describe how the reading so far is shaping your understanding. You can describe some memorable topics, describe how preconceptions are being broken down, articulate how what you are learning is integrating with your life/conversations, etc. 

This is a rather not profound way to start off, but the combination of the Keller book and Craig book has caused me to think a lot about my overall perspective on Christianity. 

A lot of the content isn't necessarily new. I have in an informal sense been interested and studied apologetics for quite some time, but what I haven't done is seen a "textbook" approach laid out combined with the "layman" approach.

With respect to how this is shaping my understanding, the readings cumulatively have done two key things. First, they are reinforcing my intellectual understanding of Christianities "correctness" and affirming the fact that believing in Christianity is actually a reasonable thing. I've continued to learn more and more about this aspect.

The second part is more interesting as far as a reflection goes. It is occurring to me that the more I know intellectually and experientially the more I find parts of my life being incongruous with the necessary reaction to the gospel. This plays out in many regards: conversation/interactions with my wife, what we do as far as our time/resources, the types of things I enjoy doing and what I enjoy, and overall philosophy of life. Seeing how this reconciles (or, as is the case, doesn't reconcile...) with the increased and more specific understanding of the validity of the gospel, from the readings, is.... well, it is interesting.

Actually studying God indirectly through the apologetics reading, writing these assignments, conversation with my wife, and conversation through the cohorts group has made me even more aware of aspects to my life where I simply do not fulfill the logical outcome of such beliefs.

For example, if I believe in the Bible/Christianity, and knowing what I do know about what Jesus calls us to do, my life should be marked by things it is not currently marked by. The interactions I have. The desires I choose to follow. The things I spend my time and money and energy on.

Last, I am continually struck by the aspect to Christianity oft mentioned in apologetics of \textit{the joy in a Christian} being a noticeable and pronounced aspect to one's life, which is a testimony to the work of Grace to others. I have pondered immensely whether or not this is present in my life to the level often described in apologetic works (both Craig and Keller mention this).

\section{How do you know Christianity is the only way?}

The first part of the answer to this question is recognizing the implicit assumptions in claiming a religion \textit{cannot} be exclusive. Keller addresses common objections well, summarized to say that many of the common objections to religion in general are not as logically sound as people present them to be.\autocite[pg.5-17]{keller2008reason} A particularly interesting one to me is the idea of moralism being relative, that if one was "born in Morocco, you wouldn't even be a Christian, but a Muslim"\autocite[pg.11]{keller2008reason} which when paralleled to pluralism, shows that any strength of the argument to pluralism is similarly as weak as its objection to religion.

I found Keller's case in chapter 1 (There can't be just one true religion) to be less of an affirmation to Christianity and more of a negation of common objections. That being said, from my perspective there are still several pieces to the answer which are introduced in Keller's writing.

First, the massive influence that early Christians had in society. Keller describes several of these, namely the increased equality and value provided to both men and women and the selflessness that Christians had within the time immediately following Jesus' life\autocite[pg.21]{keller2008reason}. Combined with the significant persecution early Christians suffered and it seems illogical to me that Christianity would not be true.

Second, the authenticity of the Bible, which will be discussed in more details in the next section. But for Christianity to \textit{not} be the only way, you must reconcile the Bible, which is perfectly clear that Jesus is the only way to God. 

\section{Why do you take the Bible literally?}

One of the most compelling arguments in my opinion to take the Bible literally is the fact that the context of the culture it was written in made it incredibly dangerous to be a Christian. The culture was not an encouraging culture for Christianity (nor was it for many years). Additionally, those in power at the time seem to have never written counterarguments - if for example, Christ hadn't risen from the dead, why did Jewish leaders not record this? The fact that there are significant claims made in the Bible which by themselves would seem miraculous have no real counterexamples or refutation seems highly unlikely, were those claims false.

Additionally, in spite of  persecution, Christianity flourished and has survived for millenia. If the Bible is not taken literally then there needs to be an explanation for this.

There are numerous literary aspects to the gospel accounts which suggest truth as well, such as those Keller discusses. Whether the detail\autocite[pg.110-111]{keller2008reason} or timing\autocite[pg.104]{keller2008reason} there are considerable pieces of historical investigation which need disproving, in order to make the case that the Bible is \textit{not} a truthful and literal source.

Another correlatory reason is the self-referencing done by the texts. Jesus, via the gospels, and the New Testament epistles often reference other scripture. While this is not a stand alone proof, it does show that the overall corpus of scripture was taken seriously by both Jesus and those NT epistle writers.

\section{What is the dance of God? How have you seen this dance play out in your own personal and/or ministry experience}

Simply put, the dance of God refers to:\autocite[pg.224]{keller2008reason}

\begin{quotation}
Each person of the Trinity loves, adores, defers to, and rejoices in the others. That creates a dynamic, pulsating dance of joy and love.
\end{quotation}

What this means practically is the interactions between God the Father, Jesus, and the Holy Spirit results in unbelievable interactions and we have been invited into this dance. Creation was created in order for us to share in it - with God! Of course, the Fall has tarnished this dance and for many years the dance was broken. But through Christ we can now be part of the dance while on Earth but more importantly, look forward to the ultimate reconciliation when we are in heaven. A glorious day!

In my life, this looks decidedly less glamorous, primarily due to sin. Reading this section actually reminds me of how self-negative I can be when it comes to being aware of my sinful nature. This translates into seeing the "dance" as being "for other people" and/or a focus on my sin, rather than the forgiveness of God and the glory of his creation and the trinity.

I suspect this is related to what I opened this assignment with: reflection regarding whether or not there is an overwhelming sense of joy and outwardly visible transformation in my life. 

\section{Presentation of Lingering Questions}

1. Are there personality types which are more inclined to the "joy" aspect of Christianity or is this something which is universally a "Christian thing?"

A common theme in this assignment within the context of reflection is my personal understanding of my joy, and more deeply, assurance of salvation in light of missing this "obvious" joy. I am unsure as to the importance this is on actual salvation but the more I read of apologetics literature describing this as part of the compelling example and evidence of the truth to Christianity the more curious I become.

2. How does affirming Christianity negate other religious worldviews and does it need to?

Keller talks about arguments that suggest exclusiveness of religion is not an intellectual problem. But in this discussion\autocite[pg.11-13]{keller2008reason} he never really addresses how significant the negation of these other worldviews actually is for the affirmation of Christianity. I suspect it is because the target audience of his book is people who are more likely to make the arguments against an exclusive religion, period, but I wonder how important defending against other religious worldviews is for purposes of Christian apologetics.

3. Do the inherent failings in eyewitness testimony present problems for the "eyewitness" approach to validating the Bible's authenticity?

Much of what Keller says in Chapter 7 is regarding the eyewitness accounts of many people affirming the Bible (or, others not rebuffing the claims early Christians made). However in modern days there is a fair bit of concern regarding the validity of eyewitness testimony, particularly as applies to the legal system. It makes me wonder whether or not the same concerns need apply to this affirmation of the Bible, and if so, how.

\newpage
\printbibliography
\end{document}

% Part I: Guided Reflection 
% Part II: Guided Discussion Topics (~1 page per question, around 3 pages total)
% How do you know Christianity is the only way? 
% Why do you take the Bible literally?
% What is the dance of God? How have you seen this dance play out in your own personal and/or ministry experience?
% Part III: Presentation of Lingering Questions (~1 page)
