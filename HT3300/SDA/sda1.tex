\begin{filecontents}{test.bib}
@book{craig2008reasonable,
  title={Reasonable faith: Christian truth and apologetics},
  author={Craig, William Lane},
  year={2008},
  publisher={Crossway}
}
@book{rath2007strengthsfinder,
  title={StrengthsFinder 2.0},
  author={Rath, T.},
  isbn={9781595620156},
  lccn={2006938575},
  url={https://books.google.com/books?id=gttDCwAAQBAJ},
  year={2007},
  publisher={Gallup Press}
}
\end{filecontents}



\documentclass[12pt]{turabian-researchpaper}


\usepackage{biblatex-chicago}

\renewcommand{\thesection}{\Roman{section}}
 \renewcommand{\thesubsection}{\thesection.\Roman{subsection}}



\renewcommand{\bibsetup}{\singlespacing}
\renewcommand{\bibitemsep}{1\baselineskip}
\renewcommand{\bibhang}{0.5in}	

\addbibresource{test.bib} % Your bibliography file



\begin{document}
\begin{singlespace}
\noindent Alden Peterson \newline
\noindent HT3300 - INTRO TO APOLOGETICS \newline
\noindent SDA 1 \newline
\noindent\rule{4cm}{0.4pt}
\end{singlespace}

\section{Guided Reflection}
% In 1-2 pages, describe how the reading so far is shaping your understanding. You can describe some memorable topics, describe how preconceptions are being broken down, articulate how what you are learning is integrating with your life/conversations, etc. 


By far and away, my biggest intellectual difficulty with Christianity has been the problem of reason. Specifically, reconciling faith vs reason. The problem I have dealt with is as follows:

\begin{itemize}
  \item If I believe in Christianity because it is reasonable or logical, then am I not entrusting a deeper faith in \textit{my reasoning} rather than Christianity?
\end{itemize}

I am a deeply logical and rational person, or at the very least, I believe so \footnote{Whether or not I actually am as logical and rational as I believe is probably outside the scope of this assignment...}. I also am confident that Christianity is \textit{plausible} or, as Craig chooses to say, that it can be shown to be true. However, this results in a question of what my Christian belief fundamentally based off of - is it my reason? Or the Truths inherent in Christianity?

If my "faith" is based on my reason alone, then my faith is not really faith in Truths at all but rather a trust in my reason. This becomes fairly difficult to reconcile with that of the New Testament focus on faith, as well as elevating my own reason to a position rivaling that of the role of faith and the Holy Spirit's work in my life bringing me to Christ.

Fundamentally, the approach Craig uses as his "Assessment" \autocite[pg.8]{craig2008reasonable} category has changed how I view this. While I am still mulling over whether this (the separation of \textit{knowing} vs \textit{showing} Christianity to be true) it seems a fairly clear distinction. The work of the Spirit allows me to know Christianity to be true while my reason/logic is the secondary aspect of showing it to be true.

In addition to this discussion, it was interesting to read through the theological implications of philosophers I have read or studied in alternative contexts. Specifically his discussion on Fyodor Dostoyevsky\autocite[pg.68]{craig2008reasonable} and Crime and Punishment. When I was considerably younger, I had read this novel however it was prior to my coming to faith. At that time I did not necessarily see it having such theological implications. Same with Ayn Rand (I have read many of her books as well).




\section*{How do you know Christianity is true?}
\section{Which argument for the absurdity of life without God did you find most compelling or convincing? Why?}
I have always been fairly confused by people who claim morality of any sort yet similarly deny the existence of God or higher powers.  It has always been a very straightforward conclusion and logical conclusion that if you deny God (or, more generally, any higher powers) that it is a necessary conclusion that morality has \textit{no basis in anything other than your own whims.} For example, Craig points out that even renowned atheists such as Richard Dawkins attempt their own moral system, such as his amended "10 commandments"\autocite[pg.81]{craig2008reasonable} approach Dawkins took. To me, this is a blatant hypocrisy and quite compelling evidence that atheism is not intellectually self-consistent.

I have always had a firm belief in objective morals, that things can be Right or Wrong in a non-preferential level. It has always surprised me how casually atheists and secular humanists dismiss the intellectual dishonesty in attempting to promote their morals as more right than someone else's, when their entire foundational position necessitates that such morals do not exist!

Craig's example of abusing children is a good, tangible example of this \autocite[pg. 87-88]{craig2008reasonable}. I find the same to be true in my own life, whenever I partake in a thought experiment regarding objective morals without a god\footnote{It is important to clarify that while Christianity is one such option in this, from the reasonableness perspective only any god(s) must exist} the atheist position is untenable. It is trivial to come up with examples an atheist would denounce as wrong, yet, there is no position from which they reasonably can do so. Things such as the Holocaust or genocide simply cannot be categorically denounced, because eventually the question of "why is this wrong?" leads to a preference or otherwise arbitrary perspective. If all morals are relative, then \textit{all morals are relative} - even those you do not find agreeable.

I should add the importance of this argument is not in validating Christianity but rather making the position of a godless existence seem absurd. Craig points this out as it is a and it is an affirmation of how this has been compelling to me personally, in validating the importance of finding an answer to the "atheism is not true" problem.



\section{How have you seen the two types of apologetics play out in your own experience? Which approach resonates most with you?}

Both types of apologetics are meaningful to me personally. Most generally though, defensive apologetics has resonated with me most significantly. I think this is for several reasons.

First, many offensive Christian apologetics are fairly weak. I have read many materials affirming Christianity which are clearly not geared towards a serious intellectual approach. I find a poor case for Christianity to be considerably worse for my faith than none whatsoever. My default reaction to a poor apologetic argument is often, "well, if this is really the best you have, how can Christianity possibly be true?" This has caused me to inherently be cynical regarding many offensive apologetic arguments.

Second, my brain is wired to be a problem solver and analytical. This causes me to simply prefer the defensive approach. Formulating a cohesive worldview is particularly important to me. Several years ago, I took a personality profile test called Strengthsfinder \autocite{rath2007strengthsfinder}, which put Belief as my highest value. This is the description for Belief:

\begin{quotation}
People strong in the Belief theme have certain core values that are unchanging. Out
of these values emerges a defined purpose for their life.
\end{quotation}

This personality trait of mine has caused me to always feel an importance to invalidating philosophical positions contrary to that of Christianity. Even so far as choosing to read literature such as the God Delusion by Dawkins. Negating such perspectives is thus of critical importance to me, as I attempt to maintain a cohesive belief system in the face of potential problem or detracting ideas.

\newpage
\printbibliography

\end{document}

% Notes: rationalizing "how do you know Christianity is true?" vs knowing
% Some people could know factually, since Jesus cannot show to me his resurrection God sends the Holy Spirit 
% is conviction part of "knowing" God exists?
% people who are non-believers are still showing signs of God's grace in their lives and some evidence of conviction by the Spirit
% offensive apologetics can lead people to Christ (and is somewhat required for that?)
% offensive is important b/c tons more defensive things

% lingering question: how are the self-evident knowing from Christianity true?
% "moment" experiences vs "lifetime" experiences?
% how do secular humanists define their morality?