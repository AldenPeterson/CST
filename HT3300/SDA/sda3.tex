\documentclass[12pt]{turabian-researchpaper}
\usepackage{biblatex-chicago}
\renewcommand{\thesection}{\Roman{section}}
\renewcommand{\thesubsection}{\thesection.\Roman{subsection}}

\renewcommand{\bibsetup}{\singlespacing}
\renewcommand{\bibitemsep}{1\baselineskip}
\renewcommand{\bibhang}{0.5in}	

\addbibresource{sda.bib}


\begin{document}


\begin{singlespace}
\noindent Alden Peterson \newline
\noindent HT3300 - INTRO TO APOLOGETICS \newline
\noindent SDA 3 \newline
\noindent\rule{4cm}{0.4pt}
\end{singlespace}

\section{Guided Reflection}
% In 1-2 pages, describe how the reading so far is shaping your understanding. You can describe some memorable topics, describe how preconceptions are being broken down, articulate how what you are learning is integrating with your life/conversations, etc. 

This reading, while particularly dense, was one I found particularly enjoyable. Having an engineering background as well as having read some content on the subject gave the astrophysical and cosmological discussion an added depth.

However by far and away the biggest personal takeaway the two chapters conveyed was being reminded of the enormity of the universe. The scale of nearly everything related to the universe is incomprehensible. The quantities (for things like numbers of galaxies or number of stars), distances (many distances are measured in light years, which by itself is nearly incomprehensible), sizes (stars are \textit{massive}), or any of the other ways to describe the universe are simply insane.

Reading sections discussing this re-reminded me of just how absurdly complicated, large, and detailed God's creation actually is in its entirety. It's easy to forget that God created not just Earth, our solar system, or even our galaxy, but \textit{all} of the universe. If you look even at your immediate surroundings it can be overwhelming; the scope of the universe is impossible to comprehend.

The enormity of this scale to me, at least as a Christian, puts me into an awe of creation - while Craig doesn't really talk about this in the two chapters for this assignment, it does somewhat help build the "knowing God to be true" case from earlier in his work.

In addition to this, I had an interesting conversation electronically recently on the subject of objective moral standards. I actually like how Craig explicitly does not call them "absolute" but instead refers to them as "objective" because in the conversation I mention, I did not do this and instead focused more on the absolute aspect.\autocite[pg.194]{craig2008reasonable} That conversation was with someone who is very much a humanist and I think a more concrete set of logic may have benefited me in handling it more effectively.

It still is astounding to me the hoops people will jump through intellectually to avoid having to admit they believe in some level of objective moral standards. I am continually amazed how inconsistent people are with respect to this, Craig even points this out within the atheistic philosopher community as well. Every defense of "some people's morals are more wrong than other people's morals" I have read seems like a major cop-out.

\section{Which argument for the existence of God do you find most compelling? Why?}


\section{Take the argument you chose above and write out how you would explain this concept to a skeptic in plain language.}
% Remember...~1 page...that's all you get :)



\section{Why should we bother taking time to understand the arguments for the existence of God?}

\section{Presentation of Lingering Questions}

\noindent 1. Is it more important to prove the existence of God or purely objective morals?

Some of the apologetic arguments argue for an existence of God and some are more from the perspective of morals. In my experience these are very much interconnected but the actual logical arguments are separate as presented by Craig.

It's not clear to me which is more "significant" (if any) from an apologetic perspective.

\noindent 2. Is the primary goal in these sorts of apologetical arguments to defend \textit{any} god(s) existence or actually to defend the Christian God?

Craig briefly talks about this when he describes the fact that Muslim apologists have thanked him for his \textit{kalam} argument\autocite[pg.193]{craig2008reasonable}. This leads me to believe these arguments do not inherently point to the existence of the Christian God/trinity but rather a god. 

It strikes me that for many people this is their primary objection to religion, such as atheists or even agnostic people. But ultimately these arguments would not seem useful in any arena other than answering, "the existence of God."

\noindent 3. Is there a significant implication between proving the existence of a singular god vs the possibility of many gods?

Many if not most of the arguments presented in this reading do not require a single god in order to be valid. For example, the causation of creation could be equally answered with a cohort of gods acting together vs a singular god. While both these perspectives are beneficial for refuting a statement, "no god exists" only one of them can affirm the Christian God and leads to the question of whether or not a followup argument for the existence of a \textit{single} god must be made in order for Christianity to logically follow.

\newpage
\printbibliography
\end{document}

