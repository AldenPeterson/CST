\documentclass[12pt]{turabian-researchpaper}
\usepackage{biblatex-chicago}
\renewcommand{\thesection}{\Roman{section}}
\renewcommand{\thesubsection}{\thesection.\Roman{subsection}}

\renewcommand{\bibsetup}{\singlespacing}
\renewcommand{\bibitemsep}{1\baselineskip}
\renewcommand{\bibhang}{0.5in}	

\addbibresource{sda.bib}


\begin{document}


\begin{singlespace}
\noindent Alden Peterson \newline
\noindent HT3300 - INTRO TO APOLOGETICS \newline
\noindent SDA 3 \newline
\noindent\rule{4cm}{0.4pt}
\end{singlespace}

\section{Guided Reflection}
% In 1-2 pages, describe how the reading so far is shaping your understanding. You can describe some memorable topics, describe how preconceptions are being broken down, articulate how what you are learning is integrating with your life/conversations, etc. 

This reading, while particularly dense, was one I found particularly enjoyable. Having an engineering background as well as having read some content on the subject gave the astrophysical and cosmological discussion an added depth.

However by far and away the biggest personal takeaway the two chapters conveyed was being reminded of the enormity of the universe. The scale of nearly everything related to the universe is incomprehensible. The quantities (for things like numbers of galaxies or number of stars), distances (many distances are measured in light years, which by itself is nearly incomprehensible), sizes (stars are \textit{massive}), or any of the other ways to describe the universe are simply insane. The numbers are of a scale we cannot comprehend.

Reading sections discussing this re-reminded me of just how absurdly complicated, large, and detailed God's creation actually is in its entirety. It's easy to forget that God created not just Earth, our solar system, or even our galaxy, but \textit{all} of the universe. If you look even at your immediate surroundings it can be overwhelming; the scope of the universe is impossible to comprehend.

The enormity of this scale to me, at least as a Christian, puts me into an awe of creation - while Craig doesn't really talk about this in the two chapters for this assignment, it does somewhat help build the "knowing God to be true" case from earlier in his work.

In addition to this, I had an interesting conversation electronically recently on the subject of objective moral standards. I actually like how Craig explicitly does not call them "absolute" but instead refers to them as "objective" because in the conversation I mention, I did not do this and instead focused more on the absolute aspect.\autocite[pg.194]{craig2008reasonable} That conversation was with someone who is very much a humanist and I think a more concrete set of logic may have benefited me in handling it more effectively.

It still is astounding to me the hoops people will jump through intellectually to avoid having to admit they believe in some level of objective moral standards. I am continually amazed how inconsistent people are with respect to this, Craig even points this out within the atheistic philosopher community as well. Every defense of "some people's morals are more wrong than other people's morals" I have read seems like a major cop-out.

\section{Which argument for the existence of God do you find most compelling? Why?}

I find the moral argument most compelling. In many ways, it seems self evident to me that there are some number of objective morals that exist. The practical details of a real situation seem to make it nearly obvious.

Questions like, "is mass murder wrong?" or "is it wrong for a parent to beat their children to death?" seem to be so blatantly obvious questions. However, the only way that they can be blatantly obvious (as well as universal) is the idea of those actions being wrong, \textit{by some external standard}. The only way I can say that Stalin's actions in the USSR were wrong is if they were wrong by comparison to a higher moral standard. It is possible to claim they were wrong \textit{according to your own moral system} but this argument can be refuted by providing situations of increasingly "obvious badness" and identifying a point when someone else's moral system is wrong enough that you must take action.

Because of this, to deny the moral argument's outcome, you must logically believe:

\begin{itemize}
\item Your moral standards are no more correct than anyone else's
\item If you attempt to impress your moral standards on someone else, you are committing an inherently immoral act by your own moral system
\end{itemize}

The second point is one I find people really dislike when practical examples are brought up in conversation. It is very easy to assent to belief in the first. However, the reality is that people actively violate their belief in it when any sort of "obvious bad" example comes up \autocite[pg.180-181]{craig2008reasonable}. For me, when an argument's strongest opposing argument is so easily refuted it is hard to not see it as compelling.

I should add that while I do find the moral argument the most compelling I do also find the kalam argument to be nearly as compelling.

\section{Take the argument you chose above and write out how you would explain this concept to a skeptic in plain language.}
% Remember...~1 page...that's all you get :)

The way I would approach this with a skeptic is to discuss whether they believe absolute morals exist. As part of conversation around this subject, I would present a situation where it is quite difficult to disagree - such as the napalming babies example from the text. 

It would be worthwhile to discuss how if absolute morals do not exist, the controversial situation cannot be condemned as categorically bad. Talking through the implications of this is a key step. Does the skeptic still feel that the action is categorically wrong? Why/why not?

Moving on from there requires discussion about what the actual implications of culturally relevant morality is. What defines that? Does majority rule? What if the majority believe napalming babies is acceptable? What if  you are not part of that culture? How does this affect the morality of the action?

After talking through this, the last step is asking if there are actions the skeptic feels strongly enough about that they would actually intervene (in another culture/country). If that conversation leads to "yes" again, the discussion should shift to be about the "why?"

Ultimately, no matter where the skeptic is coming from, the goal is to lead to the question of "why is an action actually wrong?" and leading to the belief that something/someone (God) is what adds the morality to that action.

\section{Why should we bother taking time to understand the arguments for the existence of God?}

First, it is important to affirm the intellectualness and reasonableness of Christianity in a culture where belief in religion is increasingly becoming associated with a lack of education. Being able to articulate reasons for the existence of God is consequentially critical if Christians (or, more broadly speaking, theists in general) are to receive cultural respect for even believing in God.

Second, cultural changes are resulting in many people either do not considering the question of whether God exists or simply throwing it out as hogwash. While a fundamental assumption of "yes, God exists" may have once been common it is increasingly becoming less common. Being able to answer or at the very least speak intelligently around the reasons belief in God is not intellectual dishonesty is critical to being taken seriously in an environment where the fundamental question regarding existence of God is no longer a shared belief.

Third, every person we engage with will be having a different set of assumptions, presuppositions, and experiences. The existence of God is at its core a question which underlies a lot of evangelistic difficulties. It is unlikely who does not believe in God (especially if the reason is active disbelief rather than apathy/lack of thought) will suddenly think Christianity rational.

Last, and perhaps most importantly, the question of "does God exist?" is perhaps the most significant philosophical question underlying our existence that we as humans will ever grapple with or try to understand. Ignoring it's apologetical benefit, understanding why (or, why not?) God exists has significant implications on our lives, particularly those of us who profess belief in religion. Understanding this fundamental question is thus important to confirming the validity of living a life following after the Christian worldview.


\section{Presentation of Lingering Questions}

\noindent 1. Is it more important to prove the existence of God or purely objective morals?

Some of the apologetic arguments argue for an existence of God and some are more from the perspective of morals. In my experience these are very much interconnected but the actual logical arguments are separate as presented by Craig. It's not clear to me which is more "significant" (if any) from an apologetic perspective. Or if this distinction even matters.

\noindent 2. Is the primary goal in these sorts of apologetical arguments to defend \textit{any} god(s) existence or actually to defend the Christian God?

Craig briefly talks about this when he describes the fact that Muslim apologists have thanked him for his \textit{kalam} argument\autocite[pg.193]{craig2008reasonable}. This leads me to believe these arguments do not inherently point to the existence of the Christian God/trinity but rather a god. 

It strikes me that for many people this is their primary objection to religion, such as atheists or even agnostic people to a lesser degree. But ultimately these arguments would not seem useful in any arena other than answering, "the existence of God," and pointing one to Christianity.

\noindent 3. Is there a significant implication between proving the existence of a singular god vs the possibility of many gods?

Many if not most of the arguments presented in this reading do not require a single god in order to be valid. For example, the causation of creation could be equally answered with a cohort of gods acting together vs a singular god. While both these perspectives are beneficial for refuting the statement "no god exists" only one of them can affirm the Christian God and leads to the question of whether or not a followup argument for the existence of a \textit{single} god must be made in order for Christianity to logically follow.

\newpage
\printbibliography
\end{document}

