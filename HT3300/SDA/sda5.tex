\documentclass[12pt]{turabian-researchpaper}
\usepackage{biblatex-chicago}
\renewcommand{\thesection}{\Roman{section}}
\renewcommand{\thesubsection}{\thesection.\Roman{subsection}}

\renewcommand{\bibsetup}{\singlespacing}
\renewcommand{\bibitemsep}{1\baselineskip}
\renewcommand{\bibhang}{0.5in}	

\addbibresource{sda.bib}


\begin{document}


\begin{singlespace}
\noindent Alden Peterson \newline
\noindent HT3300 - INTRO TO APOLOGETICS \newline
\noindent SDA 5 \newline
\noindent\rule{4cm}{0.4pt}
\end{singlespace}

\section{Guided Reflection}
% In 1-2 pages, describe how the reading so far is shaping your understanding. You can describe some memorable topics, describe how preconceptions are being broken down, articulate how what you are learning is integrating with your life/conversations, etc. 

I really enjoyed reading through the Keller reading for this section. I find his writing style very easy to consume, which results in it just resonating with me.

Some of the things which strike me from his reading are his accounts from conversations he has with other folks at his church and the excerpts he quotes. It makes sense to me reading through the chapters and I find them very easy and enjoyable to read.

Perhaps the most useful parts of the texts are these sections - when talking through "how I talked with people at my church" it makes it easy to relate with, both from a personal level as well as a "how would I approach this with others" way. His overall writing style contextualizes the far more academic Craig text.

I particularly liked the "Problem of Sin" section (Chapter 10\autocite[pg.165]{keller2008reason}) and  the quotes from there, regarding what people live for - prior to reading this book I am unsure I would have seen that element of the human struggle as an apologetic argument. Seeing both in myself and others the continuous striving to "justify himself"\autocite[pg.167]{keller2008reason} as confirmation of sin struck me as a unique way to approach this relatively common belief.

In addition, he goes on to say "A life not centered on God leads to emptiness\autocite[pg.173]{keller2008reason}". This is something which hit home hard for me given that my tendency is to focus on things such as career or hobbies and miss God. Or more clearly, I find myself forgetting about God - and ultimately and unsurprisingly this leads me to an emptiness which I find fulfillment only when I turn back to God or do actions which reflect the beliefs I have (such as serving, etc).

\section{The "Problem of Sin" is listed in Keller's book within the section on "The Reasons for Faith" Why? }


The primary reason the problem of sin reflects an apologetic argument is its importance and impact on the world in a \textit{non-Christian world}. On pages 175-176\autocite{keller2008reason}, Keller talks about how Jonathan Edwards discusses the impact of possible ultimate/highest goals on our society. For example:

\begin{quote}
If our ultimate goal in life is our own individual happiness, then we will put our own economic and power interests ahead of others. Edwards concludes that only if God is our \textit{summum bonum} our ultimate good and life center, will we find our heart drawn out not only to people of all families, races, and classes, but to the whole world in general.
\end{quote}

When combined with the human desire for identity, and Kirkegaard's definition of sin as anything other than having no gods before God,  it means that the only way that one can really find identity is either sin (and its resulting consequences, as defined above) or in God.

Because most people are loathe to really agree with this  - that their lives exist purely for pursuit of happiness (or some more tangible goal, like that of successful children) - it places them into a conundrum. This is made worse, because if that goal "fails" you there's... little recourse to your life purpose.  However, with Jesus, there

\section{Choose one of the following concepts from Craig and describe why that concept is important to the Christian Worldview: The Problem of Historical Knowledge}

%Choose one of the following concepts from Craig and describe why that concept is important to the Christian Worldview
%Option A: The Problem of Historical Knowledge
%Option B: The Self-Understanding of Jesus

As our society progresses towards relativism, the defense of the historical knowledge and how we obtain it becomes more and more important to understand the reasons behind trusting in the Bible's claims from a historical perspective. As Craig puts it \autocite[pg.240]{craig2008reasonable}:

\begin{quote}
Therefore, we can conclude that neither the supposed problem of lack of direct access to the past nor the supposed problem of the lack of neutrality can prevent us from learning something from history. And if Christianity's  claims to be a religion rooted in history are true, then history may lead us to a knowledge of God himself.
\end{quote}

For us as believers, it is not as important to use this as an evangelism tool but more an apologetic/philosophical viewpoint.

This applies to us as believers in two ways. First, it means we have to understand at a basic level that the Bible is trustworthy as a historical document and from a philosophical perspective as well. Second, it means that we should have some basic understanding and response to questions about the reliability of the Bible as a historical document. As Craig notes, this isn't as much an evangelism tool as it is a way to strengthen one's faith.

\section{How does the resurrection compel you toward faith in Christ?}

\section{Part III: Presentation of Lingering Questions}

1. The more I read Craig, the more confused I get - is this even bad?

I've found that the Craig text is \textit{really} hard to get through. I find it overwhelmingly academic and dense and while he straight up admits that some of the chapters (5/6 in particular) are not really useful for a practical apologetic argument,it still is somewhat disheartening to find it so dense. Alternatively, it might be the result of trying to understand an entire life's work of apologetic arguments within a semester...

\noindent 2. What do I actually think about miracles?

I do not know if I have ever really thought about miracles to the level that Craig discusses in Chapter 6. He really goes at length into this topic in a fashion I have never really considered before. I'm not really sure if this is actionable or not in my life but it definitely is something I have not really considered before.

\noindent 3. Am I at a saturation point with the material for this course and if so, what implications are there on my personal life?

One thing I've considered, particularly when reading this section, is whether or not I am simply overwhelmed with "things to think about" from this course to the point I am numb to future readings. I do not think this is the case, but rather think I am more burned out on homework given this time of year. However things to consider going forward.


\newpage
\printbibliography

\end{document}
