\documentclass[12pt]{turabian-researchpaper}
\usepackage{biblatex-chicago}
\renewcommand{\thesection}{\Roman{section}}
\renewcommand{\thesubsection}{\thesection.\Roman{subsection}}

\renewcommand{\bibsetup}{\singlespacing}
\renewcommand{\bibitemsep}{1\baselineskip}
\renewcommand{\bibhang}{0.5in}	

\addbibresource{sda.bib}


\begin{document}


\begin{singlespace}
\noindent Alden Peterson \newline
\noindent HT3300 - INTRO TO APOLOGETICS \newline
\noindent SDA 4 \newline
\noindent\rule{4cm}{0.4pt}
\end{singlespace}

\section{Guided Reflection}
% In 1-2 pages, describe how the reading so far is shaping your understanding. You can describe some memorable topics, describe how preconceptions are being broken down, articulate how what you are learning is integrating with your life/conversations, etc. 

I found this reading assignment to be considerably less impactful than the previous Craig assignment. In some sense I think this is because of the order: Keller, being more down to earth and "readable" presents similar arguments to those that Craig presents in meticulous detail.

That being said, by far and away the most interesting reading of either this section or the previous Craig reading was the discussion regarding "clues" to God. Overwhelmingly prior to this entire course, I would have articulated a need for "proof" or as a faith which is rationally unshakable and airtight. The more I read these books and discuss their content the more I am realizing how this is not as important as I once thought.

Seeing a lot of examples and articulation of how I do in fact live out my life without similar unshakable belief (such as faith in an overall epistomology) and that it is in fact impossible to conclusively have such "airtightness" when it comes to ones faith. Seeing arguments for God as \textit{clues} really provides a good paradigm for thinking of this - accepting that it is impossible to 100\% prove the existence of God, but rather attempting to make belief in God reasonable via clues.

There is also a part of me that enjoys seeing more philosophical defense of the faith and God, too. In many regards, being in science/engineering for most of my life, I do understand experientially the cultural shift towards seeing religion/Christianity as naive and/or a coping mechanism. This is a relatively common thing in those areas and I would say many people I interact with in industry have that approach. Reinforcing a positive apologetic background is therefore helpful for being reminded that belief in Christianity is actually reasonable and not intellectually dishonest. 

\section{How could a good God allow suffering?}

Keller gives several reasons for why suffering is allowed by God. First, suffering is temporary. As he says\autocite[pg.32]{keller2008reason}:

\begin{quote}
The Biblical view of things is resurrection - not a future that is just a \textit{consolation} for the life we never had but a \textit{restoration} of the life you always wanted.
\end{quote}

\noindent Within the context of a life without Heaven, suffering is considerably worse. But within the context of a restored heaven and earth the eternal implications drastically outweigh the short term negatives that are wrought with suffering. Because there is a hope, suffering matters "less" (though this is, of course, only within the Christian context).

Additionally, Jesus being fully God as well as enduring the cross means that God is not unaware of suffering and its existence: Jesus endured a greater suffering than any of us can endure. This negates the "God doesn't understand suffering" argument.

Last, and perhaps most interestingly, Keller makes the point that suffering/evil may actually be evidence \textit{for} God's existence.\autocite[pg.25]{keller2008reason} The objection itself is based on a sense of fairness or equity which is broken by evil and suffering. However that sense of fairness must make an appeal to something, which Keller suggests is God himself, thus providing an argument which results in defeating itself.

\section{In Chapters 3, 4, 5, and 6, Keller lays out some common objections to the Christian faith. Which one of these objections do you find to be most difficult to resolve? Why?}

Honestly the only real objection from those chapters which I find difficult to resolve is that presented in Chapter 5 - that a loving God sends people to hell.

The biggest difficulty is reconciling justice with love. In some sense, it is easy to see hell as a "no longer forgiveable" situation which while perhaps \textit{just} is much more difficult to see as loving. For example, how many parents would willingly condemn their child to death, if they had tried over and over (for some length) to change their behavior from sin? I think very few would be willing to do so, which is a litmus test for what our viewpoint concerning justice/love are and a reflection on their interaction.

Even "diluting" hell to be "simply one's freely chosen identity apart from God on a trajectory into infinity"\autocite[pg.80]{keller2008reason} doesn't really negate the permanence of hell. Keller tries to write off this reason with the idea that we \textit{choose} hell and also choose to stay there. 

But I am... less than convinced this is an effective refutation of the argument and more an affirmation of the difficulty to reconcile the aspect of justice and love. Because it is sidestepping the important question, which is, once you are in hell, are you there permanently or is it a continuous choice on our part?

From the perspective of total depravity, it makes sense we would be choosing hell, because we are unable to choose God without him intervening. Thus if God chooses not to do this once someone is in hell, it will never be an option to leave (because that person is unable to choose). But by the same token it means that the only reason people do not end up in hell is 100\% God's effort, which amplifies the previous problem because now it is not actually a choice of ours but rather God's lack of choice.

The reason that hell is a more objectionable aspect of Christianity is that you cannot explain hell away as a temporal thing, which you can do with the generalized evil/suffering.

\section{How have you seen the "Clues of God" or the "Knowledge of God" at work in your own life and/or in ministry?}

I think, intuitively, this is nearly exactly how my personal faith has worked itself out. I find that the "Knowledge of God" is one of the more compelling arguments for the existence of God but also one which meets the most friction with people who are non-believers.

Keller writes when it comes to "young people" that he has "not found this to be the case" that they are relativistic and amoral\autocite[pg.149]{keller2008reason}. I find this to be true in my experience too - people have strong moral claims but just do not know or have any justification for \textit{why} this is the case.

Myself? I find moral absolutes of some sort to be a pretty clear argument against an atheistic approach. However I find that many people I interact with seem to want to both believe:

\begin{enumerate}
\item Their moral values can be applied outside of cultural contexts (ie napalming babies is always wrong)
\item God does not exist
\end{enumerate}

I find this a blatant contradiction yet one people often want to take. I have often not had luck when discussing this point, either, perhaps because people need to believe both of those for life to have meaning? I am unsure honestly. But I can completely empathize with Keller's chapter on this subject from my personal life but have consistently had difficulties in relating it with other people.

\section{Part III: Presentation of Lingering Questions}

\noindent 1. Does the fact that we don't "know" anything for certain matter?

One curious observation from the texts and a theme so far the entire semester has been the idea that we cannot \textbf{know} many different things. This is used often against the various epistomologies which claim things that are self-contradictory as well as in response to many of the athesist approaches. I just wonder whether or not this has any similar self-contradiction when applied to Christianity truths. Both Craig/Keller seem to deflect this when applied to Christianity.

\noindent 2. What do apologetic arguments for other religions look like?

Something I am interested in the more we study Christian apologetics is what similar arguments look like for other religions. For example, can someone write a similar book to either of the course texts for Buddhism or Islam? Does this have any actual implication to us?

\noindent 3. How important is it to have a total and 100\% grasp on all of this material for one's faith?

There is a part of my thought process that gets overwhelmed the more I learn and consider when it comes to this. In some ways, I want a simplistic faith - because it would in fact be easier. Digesting these significant philosophical perspectives is non-trivial and intellectually taxing. I wonder how important this actually is in daily life.

\newpage
\printbibliography

\end{document}
