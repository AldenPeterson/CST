\documentclass[12pt]{turabian-researchpaper}
\usepackage{biblatex-chicago}
\renewcommand{\thesection}{\Roman{section}}
\renewcommand{\thesubsection}{\thesection.\Roman{subsection}}

\renewcommand{\bibsetup}{\singlespacing}
\renewcommand{\bibitemsep}{1\baselineskip}
\renewcommand{\bibhang}{0.5in}	

\addbibresource{sda.bib}


\begin{document}


\begin{singlespace}
\noindent Alden Peterson \newline
\noindent HT3300 - INTRO TO APOLOGETICS \newline
\noindent Capstone Project \newline
\noindent\rule{4cm}{0.4pt}
\end{singlespace}

\setcounter{tocdepth}{4}

\tableofcontents
\newpage

\section{Approach to Apologetics}
% A succinct summary of your approach to Apologetics (1-2 pages) i.e. Classical, Evidentialist, Reformed, Fideist, Integrative

My approach to apologetics is very much a classical approach. From as long as I can remember, I have always had a firm belief and understanding that an absolute moral standard does exist. This standard is something which while to me is self evident, is equally as supported from the classical arguments (which I will discuss in more detail in the position papers). 

For the past ten years I have been either in school or working as an engineer as well. Classical apologetics makes a fairly significant appeal towards the more "logic" aspect of faith, which as Craig says "[he finds] that the people who resonate most with [his] apologetic work tend to be engineers."\autocite[pg.22]{craig2008reasonable} 

I am no exception to this tendency he observes as many of the arguments made from the more classical approach. For me personally the existence of God/gods can be easily and logically deduced from the existence of morals with which no one disagrees. I find it trivial to brainstorm actions or situations which are categorically \textit{wrong} and thus, belief in a higher power necessarily follows.

This undergirds my entire apologetic approach and consequentially is the framework from which my faith is based upon. It is worth noting that while the foundation of my faith is fairly classical, the primary derivation from that approach is the existence of God. 

From a more practical, day to day apologetics, I really have a mix of the rest. Observation of the impact on Christians of "something" and seeing that within my own life results in my personal experience also having weight. However, this is much more the difference Craig describes of "knowing" and "showing." From a \textit{knowing} perspective classical apologetics has minimal impact on my life. But it 100\% is the perspective which my apologetic approach is based upon.
 

\subsection{Biblical Support}

%Biblical support for your approach (1-2 pages)
% Answer the question, “Where do I find my approach to Apologetics anchored in the Scriptures?”


\section{Position Papers}
\subsection{God as the source of morality}


\begin{quotation}
\noindent Whenever you find a man who says he does not believe in a real Right and Wrong, you will find the same man going back on this a moment later. He may break his promise to you, but if you try breaking one to him he will be complaining "It's not fair" before you can say Jack Robinson.\autocite[pg.6]{lewis2001mere}
\end{quotation}

\noindent Everyone has an intrinsic believe in "fairness" or how one ought to be treated. This is something which as Lewis says in the above quotation is a belief that while one may espouse to deny, practically speaking it is something which is lived out by most.

The basic argument is as follows\autocite[pg.172]{craig2008reasonable}:

\begin{enumerate}
\item If God does not exist, objective moral values and duties do not exist.
\item Objective moral values and duties do exist.
\item Therefore God exists.
\end{enumerate}

The first point can be considered by asking the question, "what would be the basis for objective moral values if God does \textit{not} exist?" A simple thought exercise is to think of the animal kingdom. Is it moral for a hawk to kill a rabbit? Does such a question even make logical sense - not to mention the inherent difficulty in asking moral questions about things most believe to be amoral? Regardless few would argue that a predator is inherently immoral when it kills prey to eat.

However, if God does not exist, what prevents humans from having a similar lack of moral compass? Why should a person raping another person be immoral (certainly there is no outcry for this when it occurs in nature)? Most people believe humans have and should have a \textit{different} standard of morals applied to them than does the animal kingdom. The implication is that something then is different about humankind in order to cause different moral standards to apply.

One of the most common objections to this results from a secular humanistic perspective where we as humans have created our own societal moral standards, which are why there are what appears to be moral standards. However this also falls apart when examples are presented. When is there a "universal" standard - many societies over the ages have had societal constructs which humanists today condemn. By what criteria are those condemned? 

A secular humanist struggles to answer this, for the answer is simply that there subjective criteria cannot be used as justification for applying a subjective moral as categorical and objective \textit{without} appealing to a greater power such as God. Consequentially, in order for objective moral values and duties (the requirement one act upon such duties) to matter at all they must be compared to a universal standard, God and His standards.

Once the premise (1) is made, most people will assent to (2). It takes simply discussing actions and asking the question, "is it always wrong to X?" and nearly anyone who is open minded will agree of the objectiveness of a moral action. Questions around mass atrocities such as the Holocaust or genocide can be fairly universally decreed as objectively wrong. On a more personal level, abuse or torture of children - nearly no open minded individual will disagree with this. Many secular or atheistic arguments can even be used as support of both premises, such as Nietche's argument that because God does not exist western morality is undermined. \autocite{evans2014}

Following (1) and (2), God \textit{must} exist and thus the argument is complete.
 
% Below are listed some common topics in apologetics. Choose 3 of these topics and write 2-3 page position papers that both support the Christian worldview of the topic and refute common objections.
% Existence of God
% The Bible
% Resurrection
% Suffering/Evil
% Science & Faith
% Miracles
% God as the source of morality
% Be sure that your approach to each of these topics is consistent with your overall approach to Apologetics
% Each position paper should include at least two sources outside of your course textbooks. Also, only one of these additional sources can be from an online post or blog. Published books and journal articles are preferred.


\section{Cohort Feedback}

\section{Cohort Reflection}


\newpage
\printbibliography

\end{document}
