\documentclass[12pt]{turabian-researchpaper}
\usepackage{biblatex-chicago}
\renewcommand{\thesection}{\Roman{section}}
\renewcommand{\thesubsection}{\thesection.\Roman{subsection}}

\renewcommand{\bibsetup}{\singlespacing}
\renewcommand{\bibitemsep}{1\baselineskip}
\renewcommand{\bibhang}{0.5in}	

\addbibresource{sda.bib}


\begin{document}


\begin{singlespace}
\noindent Alden Peterson \newline
\noindent HT3300 - INTRO TO APOLOGETICS \newline
\noindent Capstone Project \newline
\noindent\rule{4cm}{0.4pt}
\end{singlespace}

\setcounter{tocdepth}{4}

\tableofcontents
\newpage

\section{Approach to Apologetics}
% A succinct summary of your approach to Apologetics (1-2 pages) i.e. Classical, Evidentialist, Reformed, Fideist, Integrative

My approach to apologetics is very much a classical approach. For as long as I can remember, I have always had a firm belief and understanding that an absolute moral standard does exist. This standard is something which while to me is self evident, is equally and firmly supported from the classical arguments (which I will discuss in more detail in the position papers). 

For the past ten years I have been either in school or working as an engineer. Classical apologetics makes a fairly significant appeal towards the more "logic" aspect of faith, which as Craig says "[he finds] that the people who resonate most with [his] apologetic work tend to be engineers."\autocite[pg.22]{craig2008reasonable} 

I am no exception to this tendency he observes as many of the arguments made from the classical approach tend to resonate with me. For me personally the existence of God/gods can be easily and logically deduced from the existence of morals that are effectively absolute. I find it trivial to brainstorm actions or situations which are categorically \textit{wrong} and thus, belief in a higher power necessarily follows. To me this is a clear-cut and definitive "proof" for God's existence.

This undergirds my entire apologetic approach and consequentially is the framework from which my faith is based upon. It is worth noting that while the foundation of my faith is fairly classical, the primary derivation from that approach is the existence of God. 

From a more practical, day to day apologetics, I really have a mix of the rest. Observation of the impact on Christians of "something" and seeing that within my own life results in my personal experience also having weight. I find this affirms Christianities truths within my own life. However, this is much more the difference Craig describes of "knowing" and "showing." From a \textit{knowing} perspective classical apologetics has significantly lower impact on my life. But it 100\% is the perspective which my apologetic approach is based upon.
 

\section{Biblical Support for Approach to Apologetics}
%Biblical support for your approach (1-2 pages)
% Answer the question, “Where do I find my approach to Apologetics anchored in the Scriptures?”

Two important  questions to answer when considering whether or not the Bible supports classical apologetics will be discussed here.

First is the question, "does the Bible support intellectual arguments for the faith as beneficial?" and second, "are there examples in scripture where this is shown?"

With regards to the first, it is important to examine whether or not the intellectual argument for faith and one's understanding of Christianity is appropriate. One of my favorite examples is Romans, 12:2 (all scripture from ESV):

\begin{quotation}
Do not be conformed to this world, but be transformed by the renewal of your mind, that by testing you may discern what is the will of God, what is good and acceptable and perfect.\autocite{bible2002wheaton} - Romans 12:2
\end{quotation}

This context implies an importance to understanding and using your mind in order to understand your faith and improve it. In addition, there are numerous scattered references through the New Testament epistles on the importance of sound doctrine and a logical understanding of the faith. 

Secondly is the question as to the benefit of apologetics from Biblical examples. Here we have multiple, but primarily that of the apostle Paul, who on numerous occasions can be seen to make logical arguments in support of the Christ, such as in Acts:

\begin{quotation}
And he reasoned in the synagogue every Sabbath, and tried to persuade Jews and Greeks. - Acts 18:4

And he entered the synagogue and for three months spoke boldly, reasoning and persuading them about the kingdom of God. - Acts 19:8
\end{quotation}

Clearly, Paul found there to be value in reasoning and attempting to persuade others regarding the kingdom of God. In addition to discussing with Jews, which both above verses refer to, Paul also spent time and effort using apologetics within an evangelism sense to the gentiles, such as when he discusses with the Greeks in Athens in Acts 17. 

The book of Acts is full of examples where the apostles share with others and this often involves intellectual discussion and arguments. 

\section{Position: God as the Source of Morality}


\begin{quotation}
\noindent Whenever you find a man who says he does not believe in a real Right and Wrong, you will find the same man going back on this a moment later. He may break his promise to you, but if you try breaking one to him he will be complaining "It's not fair" before you can say Jack Robinson.\autocite[pg.6]{lewis2001mere}
\end{quotation}

\noindent Everyone has an intrinsic belief in "fairness" or how one ought to be treated. This is something which as Lewis says in the above quotation is a belief that while one may espouse to deny, practically speaking it is something which is lived out by most. The basic argument is as follows\autocite[pg.172]{craig2008reasonable}:

\begin{enumerate}
\item If God does not exist, objective moral values and duties do not exist.
\item Objective moral values and duties do exist.
\item Therefore God exists.
\end{enumerate}

The first point can be considered by asking the question, "what would be the basis for objective moral values if God does \textit{not} exist?" A simple thought exercise is to think of the animal kingdom. Is it moral for a hawk to kill a rabbit? Does such a question even make logical sense - not to mention the inherent difficulty in asking moral questions about things most believe to be amoral? Regardless few would argue that a predator is inherently immoral when it kills prey to eat.

However, if God does not exist, what prevents humans from having a similar lack of moral compass? Why should a person raping another person be immoral (certainly there is no outcry for this when it occurs in nature)? Most people believe humans have and should have a \textit{different} standard of morals applied to them than does the animal kingdom. The implication is that something then is different about humankind in order to cause different moral standards to apply.

One of the most common objections to this results from a secular humanistic perspective where we as humans have created our own societal moral standards, which are why there are what appears to be moral standards. However this also falls apart when examples are presented. When is there a "universal" standard - many societies over the ages have had societal constructs which humanists today condemn. By what criteria are those condemned? 

A secular humanist struggles to answer this, for the answer is simply that there subjective criteria cannot be used as justification for applying a subjective moral as categorical and objective \textit{without} appealing to a greater power such as God. Consequentially, in order for objective moral values and duties (the requirement one act upon such duties) to matter at all they must be compared to a universal standard, God and His standards.

Once the premise (1) is made, most people will assent to (2). It takes simply discussing actions and asking the question, "is it always wrong to X?" and nearly no one will agree of the objectiveness of a moral action. Questions around mass atrocities such as the Holocaust or genocide can be fairly universally decreed as objectively wrong. On a more personal level, abuse or torture of children - nearly no one will disagree with this as an objectively wrong act and try to allow for instances or cultures where it is openly tolerated to do things like beating children. Many secular or atheistic arguments can even be used as support of both premises, such as Nietche's argument that because God does not exist western morality is undermined. \autocite{evans2014}

Following (1) and (2), God \textit{must} exist and thus the argument is complete.

\section{Position: Existence of God}

While the previous discussion on morality is one aspect for the argument for God's existence, there are also arguments for the existence of God. One of the arguments I find most compelling is the kalam cosmological argument, formulated as:\autocite[pg.111]{craig2008reasonable}

\begin{enumerate}
\item Whatever begins to exist has a cause
\item The universe began to exist
\item Therefore, the universe has a cause
\end{enumerate}

The first premise of causality is perhaps intuitively obvious but when approached from a philosophical perspective, it requires a bit more depth and detail. Craig in his 2015 lecture at the University of Birmingham describes several reasons for it to be the case: something cannot come from nothing; if it can, it should happen more often (but does not), and our experiences and scientific processes affirm the premise.\autocite{craig2015} This leads to the conclusion that premise (1) is sound. It is worth noting that Craig spends considerable time discussing potential arguments against this, which primarily fall into the category of atheists simply claiming the universe \textit{did} in fact simply spring into being from nothing, uncaused.

The next premise is that the universe began to exist. While Craig discusses a multitude of scientific explanations and theories for the creation of the universe (which support an origin) I will focus on one, that of entropy and the second law of thermodynamics. Perhaps it is because of my mechanical engineering background but this is one I find the most compelling.

This law states that entropy always increases and never decreases. This is commonly taught as part of thermodynamics, which leads to conclusions such as:\autocite{duffy1999}

\begin{quotation}
The second law also predicts the end of the universe: it implies that the universe will end in a "heat death" in which everything is at the same temperature
\end{quotation}

Because of this, and the fact that our universe is clearly \textit{not} at this point currently, the universe must have been started in an initial state with higher entropy. Otherwise, had the universe always existed for an infinite amount of time, the effect of the second law of thermodynamics would have taken place. Because it has not, it implies a beginning of the universe. For this to \textit{not} be the case implies some external to the universe entity or force which can modify the entropy within our observed universe.

Once both the first premises are established the conclusion that the universe had a cause for its existence logically follows.

\section{Position: Resurrection} 
\textit{Note that  section has considerable discussion in the Cohort Reflection section at the end of this paper after the cohort discussion.}

\begin{quotation}
\noindent Here’s the bottom line: “If Luke was so painstakingly accurate in his historical reporting,” said one book on the topic, “on what logical basis may we assume he was credulous or inaccurate in his reporting of matters that were far more important, not only to him but to others as well?\autocite{strobel2017}

\end{quotation}
One of my favorite apologetic books regarding the resurrection is The Case for Christ, by Lee Strobel, in which he talks through the historical and scriptural basis for Christ and the resurrection.\autocite{church2009case} The above quote taken from a recent blog on his website is a compelling argument - Luke being accurate (in spite of skeptic beliefs to the contrary) on so many details is something which suggests the detail with which he writes that Luke was in fact correct in describing the history of the resurrection as well.

The resurrection and its veracity is central to the Christian faith. As Craig writes, "If Jesus rose from the dead, then his claims are vindicated and our Christian hope is sure; if Jesus did not rise, our faith is is futile and we fall back into despair."\autocite[pg.333]{craig2008reasonable} 

The importance of the resurrection for the Christian faith means the testimony of the New Testament is of utmost importance. To affirm this, there are three key parts\autocite[pg.336]{craig2008reasonable}:

\begin{enumerate}
\item The authenticity of the gospels
\item The gospel text is pure
\item The reliability of the gospels
\end{enumerate} 

Strobel's discussion above is but one piece of the evidence for the authenticity of the New Testament scriptures. Many other books of the Bible can correlate this belief, such as the New Testament epistles. Even outside the canonized Bible the writings and practices of the early church also similarly affirm the trustworthiness of the gospel accounts of the resurrection. This is \textit{not} the case with the apocryphal gospels, the text of which is a commonly made argument against the authenticity of the canonized gospels, as these did not have similar status within the early church. In addition to  this there are numerous other reasons to believe the authenticity of the scriptures such as supplemental historical texts of the time and  the overall acceptance by opponents to Christianity of the resurrection having taken place.

Purity of the gospels can be established based on the massive volume of texts we have access to from that time period. In addition, in order to have corrupted the original texts, a significant number of sources would have needed to be altered, in addition to the non-scriptural references to the gospels and their text. This would have been a significant effort and would have had to been done while many of the writers were still alive.

The most difficult part of the argument relates to reliability of the gospels. In other words, were the disciples and authors of the New Testament deceived (or deceiving others) in their writings? This is one of the more common objections to the authenticity of the Bible and must be investigated closely.

To examine this, an important counterargument to any claims of unreliable gospels is the objective fact that Christianity \textit{did} start. Regardless of whether the apostles were deceived/deceivers, Christianity did start and it started explosively in an environment where there was nearly immediately significant persecution. Clearly, these early believers (many of whom were Jews and had significant religious reasons to dislike Jesus) did believe something miraculous happened and drastically changed their lives as a result. The apostles also overwhelmingly died for their beliefs, many in gruesome ways. Any argument against the reliability of the gospels must include a compelling explanation for all of these events.

Craig articulates six specific pieces of evidence in support of the empty tomb:\autocite[pg.361-369]{craig2008reasonable}

\begin{enumerate}
\item The historical reliability of the story of Jesus' burial supports the empty tomb
\item The discovery of Jesus' empty tomb is multiply attested in very early, independent sources
\item The phrase "the first day of the week" reflects ancient tradition
\item The Markan story is simple and lacks legendary development
\item The tomb was probably discovered empty by women
\item The earliest Jewish polemic presupposed the empty tomb
\end{enumerate}

The most popular objection is that of a conspiracy among the disciples to remove the body post burial. Among other reasons, the disciples would have made this story up without context of being first-century Jews (where such a resurrection story would have been less than likely) and also the story itself as told is very lackluster for a fabrication.

Unless it can be established that an alternative explanation for the origin of the Christian church, one can confidently believe in the reliability of the gospels, as well as their authenticity and purity. This makes the case for the resurrection strong and compelling.

% Below are listed some common topics in apologetics. Choose 3 of these topics and write 2-3 page position papers that both support the Christian worldview of the topic and refute common objections.
% Existence of God
% The Bible
% Resurrection
% Suffering/Evil
% Science & Faith
% Miracles
% God as the source of morality
% Be sure that your approach to each of these topics is consistent with your overall approach to Apologetics
% Each position paper should include at least two sources outside of your course textbooks. Also, only one of these additional sources can be from an online post or blog. Published books and journal articles are preferred.


\section{Cohort Reflection}

One of the primary takeaways from the cohort discussion was a better clarity around the meaning of each type of apologetic framework, especially reformed and classical. This has caused me to wonder if my original approach from a classical perspective is really the perspective that I should have (due to time constraints my paper was completed prior to the last cohort discussion).

As we discussed these differences, I began to wonder if I actually \textit{do} think that classical arguments place too much emphasis on reason. This is something which I have pondered all my life, realistically, the question of "is my faith based on reason?" vs "is my faith based on faith" and it was interesting to discuss this as a group. I am not sure that I would articulate my approach quite as much a classical approach as I did, though, it is hard to say. 

Something else we discussed was the fact that you can make similar arguments from different apologetical approaches over the same subject. The example we talked about was the argument from morality - discussing it from both a classical as well as reformed perspective. I do not think I really thought through how some of the arguments could be so similar between the different approaches though this seems obvious in hindsight.

Craig talked a lot about the difference of "practical" vs academic apologetics and we discussed this as well, in the context of the way we would actually approach apologetics conversationally. I don't think that I would really approach my interactions with others in an evangelistic sense the way I have written this paper (though interestingly I have had conversations about this paper itself from a more classical approach so perhaps that isn't true?).

As I noted on the previous section, I have really spent time rethinking how a classical approach works within the paradigm for the resurrection. In some sense I'm not really sure that it \textit{is} possible to have a purely classical approach, which makes me wonder whether or not picking that was a good choice for this paper - perhaps I was mistaken in choosing it as an argument style? I'm not entirely sure.

Something we talked a bit about is that practically the role of the types of apologetic arguments changes the "closer" someone is to faith. It is possible that someone who is "far" from faith needs more of a classical approach while someone who is already onboard with some of the general principles of Christianity (such as existence of God) may be more receptive to reformed types of discussions. But ultimately it will be up to the person to decide.

A last topic we discussed at length was the role of the Holy Spirit within apologetics and how within the reformed approach, it's a lot more emphasized than it is within the classical role. As a Christian I am unsure what to make of this, perhaps this means it is worth considering more of a reformed approach to apologetics.

% mix of different approaches, "Faith has its Reasons" good explanation of which type of apologist you are
% can frame the same arguments from different perspectives, ie reformed vs classical
% how are classical vs reformed apologetical approaches fundamentally different?
% personal vs conversational apologetics - techniques may vary 
% Do I think classical apologists place too much trust in reason?
% Are classical apologetic arguments less beneficial the "closer" someone is to faith?
% Is my approach on the resurrection actually a classical approach or is it more an evidentialistic approach - it's heavily using logic but using a large amount of evidence for it, too?
% Role of Holy Spirit within apolgetics much higher within the reformed approach
% Difficulties in finding objections to the classical arguments because their formulation is somewhat a defense against ALL objections


\newpage
\printbibliography

\end{document}
