\documentclass[12pt]{turabian-researchpaper}
\usepackage{biblatex-chicago}
\renewcommand{\thesection}{\Roman{section}}
\renewcommand{\thesubsection}{\thesection.\Roman{subsection}}

\renewcommand{\bibsetup}{\singlespacing}
\renewcommand{\bibitemsep}{1\baselineskip}
\renewcommand{\bibhang}{0.5in}	

\addbibresource{ntsurvey.bib}


\begin{document}


\begin{singlespace}
\noindent Alden Peterson \newline
\noindent SP18 - NT Survey 1 - Gospels and Acts\newline
\noindent Unit 1 - Contextualized Assignment  \newline
\noindent\rule{4cm}{0.4pt}
\end{singlespace}

% Compose an annotated bibliography (7-10 sentences/entry) of 3 articles (journals or book chapters) regarding the canon of Scripture.

\section{Systematic Theology, Chapter 3: The Canon of Scripture}

In this chapter of Systematic Theology, Grudem\autocite[pg. 60-68]{grudem2009systematic} first focuses on the empowerment given by Jesus himself to his disciples in John 14:26. This empowerment meant that when the apostles in the early church claimed authority similar to the Old Testament prophets, the authority was qualified. This authority led the early church to consider their words as equivalent to prophets and thus inclusion of the majority of the canonized New Testament followed (excluding Mark, Luke, Acts, Hebrews, and Jude). The inclusion of Mark, Luke, and Acts were inclusive of personal testimony of the apostles (Paul and Peter in particular). Hebrews and James and to a lesser extent Mark/Luke/Acts also have a self-attestation of divine origination, in that their text leads readers to be unable to doubt the divine origin. Last, Grudem points out there are no existing books which are candidates for canonization nor books in the current canon that have remotely as universal support as the current canon.

\section{The Canon of Scripture}
Originally published in Religious \& Theological Students Fellowship in 1954, Bruce\autocite[pp. 159-161]{bruce1988canon} discusses the \textit{process} by which the New Testament was canonized. In particular, he points out a common misconception is the canon was "decided" in A.D 393, rather than simply being formalization of an existing and agreed upon canon. Most of the contents of that canon were agreed upon by around A.D 175 by the Christian church at the time and the process which occurred several centuries later was not decision making so much as formalization. This early canon creation was partially a reaction to a heretical teacher named Marcion who published an edited "canon" around A.D. 140, with Luke and 10 of the Pauline epistles edited.

\section{Reasonable Faith: The Resurrection of Jesus}

Craig\autocite[pg. 335-338]{craig2008reasonable} makes significant arguments for the authenticity of the existing New Testament. He cites an 11 point argument from Paley, the first point of which is that many of the texts were quickly referred to within early writings. Other writers of the time also referred to them with respect, describing them as Scriptures or divine writings, etc. Most of the current books were also received by those who questioned the canonization of books like Hebrews, using the example of Caius around A.D.200 including the 13 Pauline epistles while discussing the authorship of Hebrews. Many of the early attacks against the Christian church also were based on the Gospels, an affirmation of their authenticity. A last point is the practice of many of the current canon being read in public similar to those of the Old Testament by the early Christian church.

\newpage
\printbibliography

\end{document}
