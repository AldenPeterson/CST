
\documentclass[12pt]{turabian-researchpaper}
\usepackage{biblatex-chicago}
\renewcommand{\thesection}{\Roman{section}}
\renewcommand{\thesubsection}{\thesection.\Roman{subsection}}

\renewcommand{\bibsetup}{\singlespacing}
\renewcommand{\bibitemsep}{1\baselineskip}
\renewcommand{\bibhang}{0.5in}	

\addbibresource{ntsurvey.bib}


\begin{document}


\begin{singlespace}
\noindent Alden Peterson \newline
\noindent SP18 - NT Survey 2 - Romans to Revelation\newline
\noindent Unit 2 - Scholar Track\newline
\noindent\rule{4cm}{0.4pt}
\end{singlespace}



%Compose a 500-750 word essay summarizing and analyzing the quotations of the OT in 1-2 Corinthians.



\section{1st Corinthians}

\subsubsection{Quotations}
In Chapter 1, Paul quotes Isaiah 29:14 (vs 19) and Jeremiah 9:24 (vs 31).  Both of these quotes are within the context of Paul espousing that wisdom comes from God and that we should not boast in ourselves, but God. This is followed in 2:9 by a quote from Isaiah 64:4 which is in a similar context, that being wisdom coming from the Spirit and our inability to grasp them ourselves.

This theme of wisdom is repeated in Chapter 3 (vs 19-20) where Paul also quotes Job 5:13 and Psalm 94:11, where Paul continues the theme of showing the futility of human wisdom as compared with that from God.

As an exclamation point to his urging to disassociate with sexual immorality in chapter 5, Paul refers to Deut 17:7 (3:13) which encourages purging of the evil person.

When attempting to discuss his lack of burdening of the Corinthian church in Chapter 9, Paul uses the Law of Moses as a justification for how he \textit{could} have expected material things from the church (but chooses not to) by quoting Deut 25:4 (9:9).

To convince the Corinthians of the danger of sexual immorality, Paul quotes Ex 32:6 and uses the outcome that fell upon Israel for their idolatry (death) to amplify the importance of sexual purity.

Paul again invokes the Old Testament to convince the readers of God's ultimate significance by quoting Psalm 24:1 in 10:26.

Paul refers to the Law in Chapter 14:21 to clarify and highlight his previously written text on tongues.

In his closing remarks, Psalm 8:6, Isaiah 22:13, Gen 2:7, Isaiah 25:8, and Hosea 13:14 are used within the context of Paul's narrative explanation of the resurrection of the dead.

\subsubsection{1st Corinthian Summary}

Within 1st Corinthians, Paul primarily uses the Old Testament in three ways: 
\begin{itemize}
\item Affirming humanities lack of wisdom
\item Justifying his statements/commands via an appeal to the OT
\item Clarifying the resurrection of the dead
\end{itemize}  

\section{2nd Corinthians}

The first quotation of the Old Testament in 2nd Corinthians is in 4:13, where Paul quotes Psalm 116:10 in order to communicate his similar faith to that of his readers.

He then quotes Isaiah 49:8 when making the appeal to the Corinthians to not receive the grace of God in vain. 

Later in this chapter, Paul quotes and references numerous OT books (6:16b-18) in a compilation and as justification for us being the living temple of God and encouraging the believers to be holy in their hearts for the fear of God (7:1).

In 8:15 (Ex. 16:18) and 9:9 (Ps 112:9) and as justification for encouragement of the church to be generous towards God, Paul makes his final quotations of the Old Testament in 2nd Corinthians. This is part of a section spanning several chapters and ultimately culminating as a request for the Corinthian church to give money for the Christians in Jerusalem.

\subsubsection{2nd Corinthians Summary}
Paul's usage in 2nd Corinthians is much less significant than 1st Corinthians but primarily follows into the second bullet point: as justification for his statements.

\section{Theory}
%Conclude by suggesting a theory for how we should understand Paul’s use of the OT in 1-2 Corinthians.

Overall, Paul primarily uses the Old Testament in two ways within the books of 1st/2nd Corinthians. First, he uses it in sections where he heavily utilizes OT quotations to clarify his points (1Cor1,3 on wisdom; 1Cor15 on resurrection; 2nd Cor 6 on our existence as temples of God). In each of these instances, Paul heavily quotes the OT within short succession on a single topic to drive home the OT roots of that topic.

The other usage is through reference to the OT as justification for his statements, where the remainder of the quotations apply. These are interspersed throughout the books as appropriate and needed and primarily serve as an exclamation point of sorts to what Paul was saying.

\section{Notes}

References were my Crossway ESV Bible, Biblegateway, and \citeauthor{rudd17}'s online compilation of the OT quotations (for convenience in identifying a comprehensive list).

\newpage


\printbibliography

\end{document}
