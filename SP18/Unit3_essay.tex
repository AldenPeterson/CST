
\documentclass[12pt]{turabian-researchpaper}
\usepackage{biblatex-chicago}
\renewcommand{\thesection}{\Roman{section}}
\renewcommand{\thesubsection}{\thesection.\Roman{subsection}}

\renewcommand{\bibsetup}{\singlespacing}
\renewcommand{\bibitemsep}{1\baselineskip}
\renewcommand{\bibhang}{0.5in}	

\addbibresource{ntsurvey.bib}


\begin{document}


\begin{singlespace}
\noindent Alden Peterson \newline
\noindent SP18 - NT Survey 1 - Gospels and Acts\newline
\noindent Unit 1 - Essay  \newline
\noindent\rule{4cm}{0.4pt}
\end{singlespace}



%Compose a 500-750 word essay summarizing the key points in the chronological development of Historical Jesus studies as presented in C. Brown's article, "Quest of the Historical Jesus" in DJG.


\section{Key Points in Chronological Development of Historical Jesus}

Prior to the 16th century, the Christian church primarily concerned itself with the quest for the \textit{theological} Jesus. However, beginning in the early 1500s Luther and Calvin began investigation into a scholarly approach to the New Testament. This was followed throughout the next few hundred years by deviation from orthodox Christianity at the time where skepticism grew. Unitarianism and deism both grew within the 1600s-1700s and along with these beliefs, a growing analysis of the historical Jesus.

Following this, in the early 1900s Albert Schweitzer published \textit{The Quest of the Historical Jesus} which began a more formally recognized quest for the Historical Jesus. This was an analysis of Jesus from a historical perspective, where Schweitzer tied Jesus' belief in eschatology to his teaching and actions.  Within his quest, Schweitzer noted three key stages.

First, he built upon work done by David Strauss, who approached the life of Jesus based on a purely rational approach and who associated many of the miraculous events described within the Gospels as created by popular belief and culture (similar to that of the deists, who also 'removed' supernatural aspects from the life of Jesus). Next, the question of whether to base the Historical Jesus off the synoptic gospels or John, or even a source such as Q.

Following the publication by Schweitzer on the topic, there have been many more investigations into the topic, particularly after World War Two and the establishment of Israel allowing archaeological evidence to supplement the Scriptural texts. More recent and key participants in this quest are NT Wright, who has written considerable works in the 20th and 21st centuries, including some regarding his perspective on who the Historical Jesus is. Likewise, the Jesus Seminar and Dominic Crossan have presented papers portraying Jesus as a sage and healer, however interpreting the eschatology that Jesus preached to be more political and sociological in nature. 

There have also been efforts to portray the Historical Jesus as a political Jesus, who was motivated politically, or as a Jew, focused primarily or even exclusively on Jewish matters.

Finally, the Catholic church has contributed much to the understanding of the Historical Jesus. First, arguing against modernism and in affirmation of the Catholic church's supremacy and authority and later as the Catholic church began to allow such things as non-Latin translation. Numerous popes contributed works attempting to define who Jesus was.


\section{Critical Analysis}
%Include as a conclusion your own critical analysis of the history of the quest.

One of the things which I found most interesting throughout this reading was the sheer number of dramatically different perspectives which have been developed to explain and understand the personhood of Jesus. I am particularly struck how may seem to "hand-wave" away various components of the Gospels, in particular supernatural events such as miracles and the resurrection, and seem to just... pretend those do not exist. For purposes of space I suspect that the authors of DJG did not include all these arguments in detail for why this is valid but regardless it seems a bit odd to me to so easily do so.

It seems to me that the most straightforward way to approach the Historical Jesus is through asking the question, "what do the Gospels say and what was the historical context around their writing/reading?" and limiting the discussion to that context.  The authors of DJG seem to suggest many theologians and philosophers do not do this, however, in particular the deistic perspective but even more contemporary investigations such as those of the Jesus Seminar.

Additionally, it is rather sad to me to see just how many different perspectives are put forth with beliefs that strike me as frankly heretical. While the authors of DJG do admit they include the breadth of work and include "embarrassing" works, it still is sad - particularly to see even modern day theologians undertaking what seems to me to be intellectually dishonest or even heretical approaches to understanding the Gospels.

As a final note, it would be interesting to understand more fully how the theologians and philosophers discussed in this chapter view the authenticity and reliability of the Gospels. It seems that underlying many of the various theological and historical implications presented is their belief in the authenticity of the Scriptures. For example, it is much easier to discount the supernatural characteristics of Jesus' ministry if one believes the entirety of the Gospels are not intended to be historical documents. 

\newpage
\printbibliography

\end{document}
