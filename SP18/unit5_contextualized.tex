
\documentclass[12pt]{turabian-researchpaper}
\usepackage{biblatex-chicago}
\renewcommand{\thesection}{\Roman{section}}
\renewcommand{\thesubsection}{\thesection.\Roman{subsection}}

\renewcommand{\bibsetup}{\singlespacing}
\renewcommand{\bibitemsep}{1\baselineskip}
\renewcommand{\bibhang}{0.5in}	

\addbibresource{ntsurvey.bib}


\begin{document}


\begin{singlespace}
\noindent Alden Peterson \newline
\noindent SP18 - NT Survey 1 - Gospels and Acts\newline
\noindent Unit 5 - Scholar Track\newline
\noindent\rule{4cm}{0.4pt}
\end{singlespace}



%One of the important skills a teacher must have is the ability to find and summarize the content of research sources for future reference. This activity allows you to practice this while demonstrating your knowledge of the Gospel of Luke.

%Compose an annotated bibliography (7-10 sentences/entry) of 4 articles (journal or book chapters) regarding the Synoptic Problem

%Compose an annotated bibliography (at least 10 sentences each) of the articles
%Make sure your bibliographic entries conform to Turabian standards

\section{Wallace: The Synoptic Problem\autocite{wallace04}}
Wallace presents the problem as the clear interdependence of the Synoptic Gospels upon each other. First, there is clear wording agreement which indicates common source. Second, the order of how the Gospels are laid out is also quite similar. Third, a high interdependence on parenthetical material (such as the phrase ``let the reader understand" which would not make sense within a oral context). The last point Wallace presents is that of Luke's preface, which I found interesting, because he points out that Luke is clear that he is aware of other narrative accounts - ``Inasmuch as many have undertaken to compile a narrative.''

Wallace also is clear in his belief of a Markan priority as he says ``the evidence seems overwhelmingly to support Markan priority." He details in his conclusions the implications on dates of the Gospels being derived backwards from that when Acts was written (if Luke was written before Acts and Mark before Luke, this places the original texts of the Gospels ~50-60AD).

\section{Just: The Synoptic Problem\autocite{just15}}


I picked this source to see what a Catholic perspective would be. The source largely agrees with the previous source on bible.com as well as what is presented in DJG, with a focus on the Q document (presenting arguments for/against it). Dr. Just focuses on a four source theory model - where Q and Mark were used by Matthew/Luke with both Matthew and Luke pulling in their own sources as well. Dr. Just also argues for Markan priority, with an interesting point that the material in both Matthew and Luke (but not in Mark) being in varying orders within their books while sharing order for the content found from Mark. He also presents an interesting point discussing canonization of Mark but not lost, with several arguments for this - namely being the oldest recorded document of the life of Christ.

Dr. Just concludes with an interesting analysis exercise using highlighters to indicate the content in each gospel based on its shared or unique content. I would find reading a text of this form particularly interesting as it would be colored purely based on the interdependence of the Gospels and likely make the Synoptic Problem much more apparent visually.

\section{Ritsman: The Synoptic ``Problem"\autocite{Ritsman18}}



Rev. Ritsman first introduces the existing attempts to solve the Synoptic Problem, such as highlighting the Markan theory and the Q document. He expresses more skepticism towards the Q document while offering a few points of objection, such as that it is a collection of isolated phrases. 

He also approaches the principles of the problem by extensively discussing the oral tradition of the culture at time. Uniquely he points out the practice of written notes to supplement the oral traditions. He also mentions that it is a likely fact that the Christian message was first taught through oral teaching, particularly to the early church and in the timeframe before the gospels were written.

Rev Rivsman addresses the question of whether the gospel authors were authors or merely editors, suggesting that at the very least Luke was an independent writer. 

Following this he lists four factors to consider when analyzing these. First, that the gospels were accepted very early within the church, meaning their authenticity must have been easily and reliably accepted in the early church. Next, that the gospels are regarding a unique person and so must therefore be unique. Third, and interestingly, that the gospel material from the written gospels were \textit{not} the primary root of early Christian activities. Last, the role of the Holy Spirit within the origins of the Gospels potentially providing context for how the gospels were written similarly.


\section{Wikipedia: The Synoptic Problem\autocite{wikipedia18}}
The last source I looked at was Wikipedia. I did this for several reasons:
\begin{itemize}
\item While not strictly an academic source, it often is a good synthesis of existing sources
\item It would likely be written more on a non-Christian perspective and I am curious what that perspective is on the problem
\item Wikipedia often is written more readably than academic texts such as DJG and the previous articles
\end{itemize}

The first unique aspect is that Wikipedia introduces a phrase I did not recognize from other sources, that of ``documentary dependence" - a generalized way to discuss interdependence of the different texts. It then discusses controversies regarding the Synoptic Problem as actually the \textit{defining} it (which was actually useful given the lack of clear definition of the "problem" in most Christian texts). Namely that the questions of priority, dependency, lost written sources (such as Q), oral sources, translation, and potential redaction within the gospel writings. While Wikipedia itself did not do a significant detailed explanation on each it was presented in a nice format.  It also had a good visual presentation of the various possible sources.

As a standalone source, it would not have been a good source. But in combination with the more academic sources I found it actually presents information in a very intelligible and visually digestible way.


\newpage


\printbibliography

\end{document}
