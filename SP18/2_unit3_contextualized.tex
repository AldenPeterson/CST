
\documentclass[12pt]{turabian-researchpaper}
\usepackage{biblatex-chicago}
\renewcommand{\thesection}{\Roman{section}}
\renewcommand{\thesubsection}{\thesection.\Roman{subsection}}

\renewcommand{\bibsetup}{\singlespacing}
\renewcommand{\bibitemsep}{1\baselineskip}
\renewcommand{\bibhang}{0.5in}	

\addbibresource{ntsurvey.bib}


\begin{document}


\begin{singlespace}
\noindent Alden Peterson \newline
\noindent SP18 - NT Survey 2 - Romans to Revelation\newline
\noindent Unit 3 - Contextualized Assignment
\newline\noindent\rule{4cm}{0.4pt}
\end{singlespace}



\section{Understanding the Law\autocite{gatewayLaw17}}

This commentary focuses on Galatians 3 where Paul introduces the question regarding the purpose of the Law (vs 19). The author suggests three primary purposes in Paul's reply. First, that the Law has a negative purpose - it provides the standard for which we can be measured. Second, the temporary nature of the Law (3:19). This temporality is the basis for Paul's arguments on 
the supremacy of Christ over the Law. The third key point relates to the origin of the Law coming from angels and Moses. Paul argues the fact that the Law came through mediators actually is a weakness (particularly compared with Christ). 

The commentary continues with some general observations. Central to the Paul's argument is also the theme of the Law being unable to \textit{provide} life. Because this is not possible, we cannot achieve a right relationship with God specifically via the Law.

Finally, a discussion on the role of the Mosaic Law is presented. Specifically the author talks through the intended temporality of the Law. They conclude with an interesting analogy of their friends who moved to North America from previously living in a dictatorship. The author points out most would fine those friends crazy if they tried to live under the tyrannical law from their former country.

\section{The Role of the Old Testament Law in Galatians\autocite{schreiner17}}

Schreiner starts off with possible interpretations of why the Law was given, based on the Greek of the letter itself. He lists common positions being that God gave the Law to either: restrain sin, define sin, deal with sin, or increase sin. Schreiner posits the first is incorrect because otherwise the Judiazers would have been correct and thus negate much of the reason Paul had to write Galatians. Defining sin becomes difficult to reconcile with Christ's teaching. Atonement is also easily negated, leaving the role of the Law to increase sin.

He then offers a few conclusions on the Law. First, that it is provincial. Similar to the previous article's focus on temporality Schreiner points out Galatians is clear that the Law was not intended to exist forever. The Law is also subservient (given through people).  Galatians 3:20 clarifies too the inferiority of the Law compared to God's promise.

Last he writes that the Law does not transform hearts. An interesting point he makes is how counter cultural the claim that the Law does not actually restrain sin but rather makes it worse would have been to the culture Paul wrote to, where Jews had a more common perspective that the Torah brought moral transformation.

\section{Paul's View of the Law in Galatians and Romans \autocite{kulikovsky99}}

This article compares Galatians and Romans (the previous two were explicitly on Galatians). Kulikovsky starts by asserting that neither book contains a formal or systematic discussion on the Law (but rather more ad hoc, directly addressing questions/issues). He begins talking through the Abrahamic covenant and observing the Law came 430 years \textit{after} this, thus the promise could not have come through the Law which did not exist.

Next Kulikovsky addresses the problem of the Law being a curse. He observes that not only in Galatians but also in Deuteronomy the Law is described this way and that it is a slavery (via Gal 4:22-31). Following this is the argument for the supremacy of Christ to the Law both by releasing us from it and fulfilling the Law.

The final argument Kulikovsky makes is to the purpose of the Law. Interestingly this is at the end of his paper, while it was a more central theme to the other two. He describes it similar to other articles, that being the Law is good and that it highlights sin and instructs us of our sin.

I found the introduction here interesting because I never really considered that Paul doesn't give a more academic discussion on the Law; always within the context of other points. It makes me realize how much inferring we do in order to really understand what Paul's overall perspective on the Law actually was.

My major critique of this article would be its brevity. It may be simply because I read more focused articles prior but it felt very high level and I think with some more fleshing out would have been a fascinating article (it was only 9 pages).
\newpage


\printbibliography

\end{document}
