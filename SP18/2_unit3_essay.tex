
\documentclass[12pt]{turabian-researchpaper}
\usepackage{biblatex-chicago}
\renewcommand{\thesection}{\Roman{section}}
\renewcommand{\thesubsection}{\thesection.\Roman{subsection}}

\renewcommand{\bibsetup}{\singlespacing}
\renewcommand{\bibitemsep}{1\baselineskip}
\renewcommand{\bibhang}{0.5in}	

\addbibresource{ntsurvey.bib}


\begin{document}


\begin{singlespace}
\noindent Alden Peterson \newline
\noindent SP18 - NT Survey 2 - Romans to Revelation\newline
\noindent Unit 3 - Essay
\newline\noindent\rule{4cm}{0.4pt}
\end{singlespace}



% Compose a 500-750 word essay (2-4 pages) analyzing the doctrine of the Holy Spirit within the argument of either Galatians or Ephesians. Be sure to cite specific scriptural evidence and the article “Holy Spirit” in DPL

\section{Organization}
Each reference within Ephesians will be briefly discussed and associated with the categorizations found within the DPL chapter on Holy Spirit\autocite[pg404-412]{hawthorne2009dictionary} (Spirit of... God, Wisdom, Divine Power, Christ, Mission, Christian's New Life, Eschatology, and Worship).


\section{Ephesians}



\begin{quote}
Ephesians 1:13-14\footnote{All quotations from ESV via biblegateway.com} -  In him you also, when you heard the word of truth, the gospel of your salvation, and believed in him, were sealed with the promised Holy Spirit, who is the guarantee of our inheritance until we acquire possession of it, to the praise of his glory.
\end{quote}




\begin{quote}
Ephesians 1:17 - that the God of our Lord Jesus Christ, the Father of glory, may give you the Spirit of wisdom and of revelation in the knowledge of him
\end{quote}

\noindent Paul's opening to Ephesians indicates a transferal of the Spirit - from God, to us. The Spirit originates from God, being the Spirit of God. Interestingly, the authors of DPL do not actually use this verse as support for their categorization. This seems to be an omission from the entirety of the chapter on the Holy Spirit.

\begin{quote}
Ephesians 2:18 - For through him [Christ] we both have access in one Spirit to the Father
\end{quote}

\noindent Chapter two makes a straightforward claim that it is \textit{through} Christ whom we have access to the Spirit (as well as the Father through the Spirit). 

\begin{quote}
Ephesians 2:22 - In him you also are being built together into a dwelling place for God by the Spirit.
\end{quote}

\noindent The active role the Spirit plays in the life of the believer and/or the church is shown here by a relatively descriptive role the Spirit plays within the context of either (however difficult the interpretation of whether the "you" refers to an individual or collective church). 

\begin{quote}
Ephesians 3:4-5 - When you read this, you can perceive my insight into the mystery of Christ,  which was not made known to the sons of men in other generations as it has now been revealed to his holy apostles and prophets by the Spirit
\end{quote}

\noindent Here, Paul confirms the origin of both his inspiration but also those of the prophets as coming from God via the Spirit. It seems evidence Paul views the Spirit as God's method of transferring information (he uses "insights") to humans, an example of the Spirit's Wisdom.

\begin{quote}
Ephesians 3:16-17 - that according to the riches of his glory he may grant you to be strengthened with power through his Spirit in your inner being so that Christ may dwell in your hearts through faith—that you, being rooted and grounded in love
\end{quote}

\noindent This indirectly shows that the Holy Spirit will, through His power, strengthen believers, and clearly falls into the category of the Spirit's Power from DPL.  Additionally it shows a secondary aspect of the association to Christ, via 3:17.

\begin{quote}
Ephesians 4:3-4 - eager to maintain the unity of the Spirit in the bond of peace. There is one body and one Spirit — just as you were called to the one hope that belongs to your call
\end{quote}

\noindent This is the first instance in Ephesians where  \textit{attributes} of the Holy Spirit are shown vs an action of the Spirit, that being an association of the Spirit to the unity of the church and the singular nature of the Spirit.

\begin{quote}
Ephesians 4:30 - And do not grieve the Holy Spirit of God, by whom you were sealed for the day of redemption
\end{quote}

\noindent Yet another reference to an action of the Spirit, 4:30 clarifies more on the "how" the Divine Power of the Spirit is manifested in our individual lives.

\begin{quote}
Ephesians 5:18-19 - And do not get drunk with wine, for that is debauchery, but be filled with the Spirit, addressing one another in psalms and hymns and spiritual songs, singing and making melody to the Lord with your heart,
\end{quote}

\noindent It seems from 5:18-19 that when we interact with other believers we are to be "filled with the Spirit" as part of our actions. The tangible outcome of this is listed in 5:19 and can be inferred to be what it looks like in the life of a believer to be worshiping the Lord.

\begin{quote}
Ephesians 6:17-18a - and take the helmet of salvation, and the sword of the Spirit, which is the word of God,  praying at all times in the Spirit, with all prayer and supplication. 
\end{quote}

\noindent This section describes two key points as relates to the Spirit. First, the interaction of the Holy Spirit and the word of God being highly related. Second, that as believers, we ought to pray "in the Spirit" affirms previous references (2:18) which describe an intermediary role the Spirit has between  us and the Father.

\section{Conclusion}

Many of the large scale categorizations found within DPL are seen through Ephesians. Interestingly, there is minimal evidence of the "Spirit and Mission" and "Spirit and Eschatology" section, though these seem to be primarily drawn from Romans and Corinthians.

I found it interesting to see that in spite of the relatively numerous references to the Holy Spirit within Ephesians, the authors of DPL did not directly associate them. They did however continue to reuse a relatively small set of references (1:13-14) throughout many of the sections.

I am unsure whether this suggests an organizational issue with how either I approached the material or how the DPL chapter itself is organized. The book of Ephesians seems to have a lot more description on what the Holy Spirit does and the interaction between God the Father and the Holy Spirit, which is suspiciously absent from the DPL chapter categorization. However in Ephesians this seems to be the primary theme within all the mentions of the Spirit.

\newpage


\printbibliography

\end{document}
