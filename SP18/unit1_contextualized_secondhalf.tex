
\documentclass[12pt]{turabian-researchpaper}
\usepackage{biblatex-chicago}
\renewcommand{\thesection}{\Roman{section}}
\renewcommand{\thesubsection}{\thesection.\Roman{subsection}}

\renewcommand{\bibsetup}{\singlespacing}
\renewcommand{\bibitemsep}{1\baselineskip}
\renewcommand{\bibhang}{0.5in}	

\addbibresource{ntsurvey.bib}


\begin{document}


\begin{singlespace}
\noindent Alden Peterson \newline
\noindent SP18 - NT Survey 2 - Romans to Revelation\newline
\noindent Unit 1 - Scholar Track\newline
\noindent\rule{4cm}{0.4pt}
\end{singlespace}




\section{Justification: The Saving Righteousness of God in Christ\autocite{Shreiner2011}}

Schreiner begins by discussing a work by N. T. Wright called The Resurrection of the Son of God and Wright's perspective on the interaction between salvation and works. He writes to cover three primary objections to this - ecclesiology vs soteriology, a false polarity with respect to the mission of Israel, and a missing imputation of God's righteousness associated with justification.

The first objection is to clarify that Paul talks at length about salvation, outside the context of the church (or our interactions with the church, e.g. circumcision, baptism, etc). Schreiner posits that Wright overly focuses on these interactions to the detriment of the exclusively soteriological perspective Paul articulates in his writings, particularly Romans.

The section on Israel is relatively short and more or less an argument that Israel \textit{itself} had something deeply wrong with it, not just its actions.

Last, he argues that God can in fact impute righteousness in the courtroom like situation we will one day find ourselves in (contrary to what Wright posits). Rather than a declaration of innocence Schreiner argues we will actually be imputed with righteousness.


\section{The Apostle Paul and the Introspective Conscience of the West\autocite{Stendahl63}}

Stendahl concerns himself with whether or not a contemporary understanding of an introspective conscience is appropriate to what Paul wrote. He writes that Paul never really addresses a conscience in his writings. He also observes that Paul was not a hugely impactful figure in the early church, prior to Augustine around 350 AD, following this to Luther. Much of this article is reiterating this point - that we (collective Western civilization) have read more into what Paul said regarding consciousness. Specifically that we read introspective consciousness rather than blaming Sin and that Paul treats the mind/will of man as on the side of God. As Stendahl says, "He [Paul] would be suspicious of a teaching and a preaching which pretended that the only door into the church was that of evermore introspective awareness of guilt and sin."

I found this article fascinating, particularly to read prior to my read through of Romans. 


\section{The Paul of History and the Apostle of Faith\autocite{wright78}}

Wright's essay here is his perspective on the previous work by Stendahl and its critiques by Käsemann.  His first clarification is making clear that both Stendahl and Käsemann have presupposed various beliefs into their analysis; Stendahl that Jews have a path to salvation exclusive of Jesus and Käsemann's trust in historical context.

His first argument is an interpretation of Paul's beliefs to be that the people of Abraham were God's ``answer" to the problem of Adam. However, these Jews were not the light to the world they were intended but a darkness and consequentially resulted in needing a similar ``answer" as they were to be for Adam. Jesus therefore is the representative of Israel and "solves" the problem of Abraham's people, Adam, and consequentially everyone on earth.

Second, he addresses the question of who the ``real Paul" is compared to common interpretation and impressions. Starting off, he raises the question of Judaism being legalistic is a more new (1600s+) belief, meaning that our understanding of the Jewish culture who opposed Paul has been tainted to believe Paul has conflict with a strawman position rather than an appropriate understanding of historical Judaism. Namely, that Jews were using their position as ``the people of God" to cloud their understanding of how they interact with God, seeing that alone as a righteousness in and of itself.

I found this difficult to find much fault in, excepting perhaps that I think he could have elaborated more on the implications of how Jesus representing fulfillment of the purpose of Israel include us as Gentiles (rather than purely those of Jewish lineage). It would have also been useful to see context for how Israel and the nations could represent God's people - if Israel was to restore God's people (all of mankind?) then what of the non-Israel nations who were at war with Israel? Abraham was not the only descendant of Adam and I don't understand why only Abraham and consequentially Israel represent fulfillment of the entirety of the ``problem'' of Adam.
\newpage


\printbibliography

\end{document}
