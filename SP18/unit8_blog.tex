
\documentclass[12pt]{turabian-researchpaper}
\usepackage{biblatex-chicago}
\renewcommand{\thesection}{\Roman{section}}
\renewcommand{\thesubsection}{\thesection.\Roman{subsection}}

\renewcommand{\bibsetup}{\singlespacing}
\renewcommand{\bibitemsep}{1\baselineskip}
\renewcommand{\bibhang}{0.5in}	

\addbibresource{ntsurvey.bib}


\begin{document}


\begin{singlespace}
\noindent Alden Peterson \newline
\noindent SP18 - NT Survey 1 - Gospels and Acts\newline
\noindent Unit 8 - Blog Assignment\footnote{Conclusions drawn primarily from coursework, readings, and knowledge of this topic I have learned over the past years. I am unsure how/where to cite the influences on this as a result}
\newline
\noindent\rule{4cm}{0.4pt}
\end{singlespace}


%Compose a 500-750 word blog entry briefly summarizing the rationale of Jesus' opponents, as identified in the articles "Pharisee," "Sadducees," and "Sanhendrin" in DJG and John 11:47-53

%Feel free to be creative, utilizing visual aids if necessary.
%Make sure you cite specific verses and any source material

\section{Opposition to Jesus}
\begin{quotation}
Mark 15:1-3: 
And as soon as it was morning, the chief priests held a consultation with the elders and scribes and the whole council. And they bound Jesus and led him away and delivered him over to Pilate.  And Pilate asked him, “Are you the King of the Jews?” And he answered him, “You have said so.”  And the chief priests accused him of many things.\footnote{All biblical quotations from ESV}
\end{quotation}

\noindent When Jesus was presented to Pilate, his enemies - called out clearly in Mark 15 - made it clear that he was accused of many things. These things can be found throughout the gospels with more specific examples.

\section{Messianic Claims}

Perhaps the most grievous claim brought against Jesus was with respect to his claims of being the Christ. This is clearly articulated within the Synoptic gospels such as in Matthew 26:63-65:

\begin{quotation}
But Jesus remained silent. And the high priest said to him, ``I adjure you by the living God, tell us if you are the Christ, the Son of God.''  Jesus said to him, ``You have said so. But I tell you, from now on you will see the Son of Man seated at the right hand of Power and coming on the clouds of heaven.”  Then the high priest tore his robes and said, ``He has uttered blasphemy. What further witnesses do we need? You have now heard his blasphemy.  What is your judgment?'' They answered, ``He deserves death.''
\end{quotation}

The reasons why Jesus was ultimately sentenced to death by the leading Jewish leaders of the time boiled down to his claims of being the Messiah. 

It's worth pointing out that the Pharisees were not expecting a "God" savior, but rather a Davidic king type of savior. Jesus, while of the line of David, was claiming to be God - this violated the expectations the Pharisees had regarding just \textit{who} Jesus would be. While other factors contributed to the religious leadership wanting to arrest Jesus, the Gospels make this specific claim from Jesus was what directly influenced their desire to see Jesus put to death.

\section{ The Law}

Previous to the claims of divinity ultimately resolving in the Passion, Jesus was often at odds with the Pharisees regarding observance of the Law (as well as Jesus's ability to interpret the Law). Examples such as Mark 3:1-6 show this dynamic:

\begin{quotation}

Again he entered the synagogue, and a man was there with a withered hand.  And they watched Jesus,[a] to see whether he would heal him on the Sabbath, so that they might accuse him. 3 And he said to the man with the withered hand, ``Come here.''  And he said to them, ``Is it lawful on the Sabbath to do good or to do harm, to save life or to kill?” But they were silent.  And he looked around at them with anger, grieved at their hardness of heart, and said to the man, ``Stretch out your hand.” He stretched it out, and his hand was restored.  The Pharisees went out and immediately held counsel with the Herodians against him, how to destroy him.
\end{quotation}

This occurs directly after Mark 2:23-28 when Jesus corrects the Pharisees in their understanding of the Sabbath and declares himself ``the Son of Man is the lord even of the sabbath.'' 

The religious leaders found it impossible to reconcile a Messiah who, to them, had a blatant disregard for the Mosiac Law.

\section{Direct confrontation with Pharisees}

Jesus also directly confronted the Pharisees, both via the ``woes'' type of conflict represented in Matthew 23 but also through parables such as Luke 18 where a sinner is told to be in better standing with God than the Pharisee. In Matthew 21 Jesus tells multiple parables which the Pharisees heard and understood to be about them:

\begin{quotation}
When the chief priests and the Pharisees heard his parables, they perceived that he was speaking about them.  And although they were seeking to arrest him, they feared the crowds, because they held him to be a prophet.\footnote{Matthew 21:45-46}
\end{quotation}
\newpage

These sorts of confrontations are scattered throughout the Gospels and further reflection of the opposition to Jesus. Jesus did not fear conflict with the religious leaders and whether through these parables, being challenged by the Jewish leaders, or simply through direct accusation engaged with the religious leadership in a way which generated significant opposition.


\printbibliography

\end{document}
