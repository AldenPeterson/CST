
\documentclass[12pt]{turabian-researchpaper}
\usepackage{biblatex-chicago}
\renewcommand{\thesection}{\Roman{section}}
\renewcommand{\thesubsection}{\thesection.\Roman{subsection}}

\renewcommand{\bibsetup}{\singlespacing}
\renewcommand{\bibitemsep}{1\baselineskip}
\renewcommand{\bibhang}{0.5in}	

\addbibresource{ntsurvey.bib}


\begin{document}


\begin{singlespace}
\noindent Alden Peterson \newline
\noindent SP18 - NT Survey 1 - Gospels and Acts\newline
\noindent Unit 5 - The Synoptic Problem\newline
\noindent\rule{4cm}{0.4pt}
\end{singlespace}


% Reflect on the article "Synoptic Problem" as presented in DJG
% Conclude by briefly arguing for a solution to this "problem"
% Compose a 500-750 word essay (2-4 pages), addressing the following:



\section{Reflection}

Prior to this week's readings, I am not sure I was ever really aware that the Synoptic gospel similarity presented a "problem" (as the chapter is titled). In fact this is one thing the author of DJG does not really do a compelling work towards - presenting the "why" for whether or not it actually is a problem for the historian or theologian.

That being said, it is interesting to reflect upon the similarities between the gospels. I am not sure whether it is my ignorance prior to the section or what, but I really never found that to be a significant problem at all for the cohesiveness or veracity of the gospels and New Testament accounts of Jesus.

One thing I really wish the author would have expanded upon is why this actually is a problem. It seems self-evident to me that it is actually more beneficial to the cause of Christianity and reliability of the NT to have the gospels actually agree (even if verbatim) because it suggests similarity in either:

\begin{itemize}
\item The gospels shared common background, whether the Q document or verbal traditions
\item The writers, with varied backgrounds themselves, still agreed upon so much text

\end{itemize}

To me both of these conclusions are actually in \textit{favor} of the reliability of the gospels rather than problematic, regardless of the actual first century reasoning. However the author does not really address this at all but instead just jumps straight into the various possibilities for the explanation.

A lot of the possible explanations seem to fall into either derivation from a common source (Q) or utilization of the gospels in some fashion by other writers, with differing opinions on this between various scholars.

It also seems careless to read through or approach the Synoptic Problem without more fully appreciating and understanding the historical context in which the gospels were written. For this reason I am thankful that this reading also included the sections describing the oral tradition at the time of the gospels. Even the authors seem to point this flaw out in some of the analysis of the Synoptic Problem (or within the Q chapter) in that the proponents of various theories treat the texts as if they were written in a post-Gutenberg context, when in fact they were written in a timeframe where our understanding of writing and history would not be the same.


\section{Solution}

Honestly, it is difficult for me to really argue for a solution when I do not fully understand the problem in the first place. However I find it pretty easy to believe that:

\begin{itemize}
\item Matthew, Luke, and Mark were written in a time where oral tradition dominated
\item It is likely the disciples memorized and/or wrote down the key teachings of Jesus
\item A common source (such as Q or the oral traditions) existed
\item We may not be able to know conclusively in our earthly life what the actual reason for the similarity in the Synoptic Gospels is
\item The veracity and reliability of the gospels is unaffected
\end{itemize}

Because of this, I suppose my "solution" would be some level of acceptance of unknowing and belief that regardless of the actual root cause for the similarities, my Christian faith will be unaffected. This is likely made easier in that I am a "casual" NT scholar and not someone who "needs" to have an answer to the question in a more academic sense.

I also think it is important to characterize this problem within the other NT books and recognizing the cohesiveness between the message and life implications presented in the Synoptic gospels with those of the epistles, Acts, and books of John. If the Synoptic gospels were alone in their themes and messages the importance of the problem would become much more immediately relevant to me, however, this does not seem to be the case.

\newpage

\printbibliography

\end{document}
