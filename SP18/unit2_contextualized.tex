\documentclass[12pt]{turabian-researchpaper}
\usepackage{biblatex-chicago}
\renewcommand{\thesection}{\Roman{section}}
\renewcommand{\thesubsection}{\thesection.\Roman{subsection}}

\renewcommand{\bibsetup}{\singlespacing}
\renewcommand{\bibitemsep}{1\baselineskip}
\renewcommand{\bibhang}{0.5in}	

\addbibresource{ntsurvey.bib}


\begin{document}


\begin{singlespace}
\noindent Alden Peterson \newline
\noindent SP18 - NT Survey 1 - Gospels and Acts\newline
\noindent Unit 2 - Contextualized Assignment (Scholar Track)  \newline
\noindent\rule{4cm}{0.4pt}
\end{singlespace}


% Compose a 750-1000 word essay (3-4 pages) that summarizes the quotations of the Pentateuch in Matthew and synthesizing them into a coherent conclusion.
% Identify the quotations of each book of the Pentateuch
% Synthesize them according to Matthew's use of each book (i.e. what themes/ideas does Matthew communicate by means of Genesis, then Exodus, then Leviticus, then Numbers, then Deuteronomy)
 
\textit{Logistical note: I found it far easier to find an external list of citations and work through  book by book with a brief summary with the text, hence this organization.}
\section{Genesis}

Matthew's quotation of Genesis comes in 19:4-5, quoting Genesis 2:24. This quotation is explicitly in response to a Pharisee questioning Jesus about the legality of divorce for any cause\footnote{19:1-9}. His response here is clarification of what God's original intent for marriage was, particularly regarding divorce, and that the reason Moses allowed divorce was "your hardness of heart" even though "from the beginning it [allowing divorce] was not so."\footnote{19:8}


\section{Exodus}

Exodus is quoted three times in the sermon on the mount - 5:21 (Ex. 20:13), 5:27 (Ex. 20:14), and 5:38 (Ex. 21:24). Each of these three is part of Jesus clarifying various commandments and teaching of the Old Testament and part of the larger sermon narrative, regarding the inability of anyone to enter the kingdom of heaven on their own righteousness.\footnote{5:20} In all three of these quotations, Jesus expands on the commands and makes them more difficult to achieve. This points to our inability to achieve righteousness on our own as well as Christ's authority to interpret scripture.

In Matthew 15:4-5 Jesus quotes Exodus 20:12 (to honor your mother/father) as well as Exodus 21:17, which is the penalty for not doing so. He does this in context of Pharisees asking him pointed questions around why he and his disciples do not follow their customs as a way to point out their hypocrisy and reference Isaiah 29:13, further amplifying the fact that the Pharisees and scribes have allowed human customs to override their respect for the Scriptures.

In Matthew 19, Jesus again quotes from the 10 commandments, this time within the context of what commands we must keep in order to have eternal life. This is done as part of a conversation with a man who has many earthly treasures and is the leadin to a larger message in Matthew 19:23-30 regarding the difficulty of the wealthy to enter the kingdom of God.

The last time Exodus is quoted is within the context of the resurrection of the dead, in talking to the Sadducees.
\footnote{22:32} He quotes Exodus 3:6 which makes the crowd "astonished at his teaching" and reflects Jesus' understanding of the Scriptures.

\section{Leviticus}

Multiple quotations from Leviticus fall within the sermon on the mount and have similar purpose as those from Exodus previously discussed. Jesus quotes Leviticus in 5:33 (Lev 19:12), 5:38 (Lev 24:19-20), and 5:43 (Lev 19:18). These again are quotations of commands given in the Pentateuch and explained more fully by Christ.

Leviticus is again quoted in parallel with Exodus in the conversation with the young rich man, when Jesus quotes Lev 19:18 (for the second time) as an example of commands the man needs to obey in order to have achieved eternal life. \footnote{19:19}

In Matthew 22:39, the same passage from Leviticus is once again used (3rd time) as Jesus explains the "greatest commandments" - "And a second is like it: 'You shall love your neighbor as yourself.'" It is interesting to note that in addition to this being used in this context as the second commandment, after loving God with all your heart/soul/mind, Matthew quotes it three times in his book, underlying the importance that Jesus and Matthew felt the passage has.

\section{Numbers}

Numbers is cited just once within Matthew, within the context of the sermon on the mount, when in Matthew 5:33 Jesus refers to Num 30:2\footnote{Interestingly my ESV Bible does not show this reference, though the parallel is very clear when looking at the text of Num 30:2 - perhaps an interesting followup question is why this was not chosen to be referenced by the ESV writers}. This falls into the same purposes as done with the Leviticus and Exodus citations.

\section{Deuteronomy}

Deuteronomy is quoted far more than the other books in the Pentateuch by Matthew, beginning in Matthew 4, when Jesus quotes it three times in response to the devil (Deut 8:3, in 4:4; Deut 6:16, in 4:7; Deut 6:13, in 4:10). These responses all come within the first acts recorded about Jesus in his adult life after his baptism, when the devil tempted him. This showed that Jesus not only knew the Old Testament but was equipped to respond to the devil via scripture. It immediately introduces Jesus as someone with an authoritative and working knowledge of the scriptures.

Deuteronomy is quoted four times within the sermon on the mount for similar reasons as those previously discussed - in 5:21 (Deut 5:17), 5:27 (Deut 5:18), 5:31 (Deut 24:1-2), and 5:38 (Deut 19:21).

In Matthew 18:16, Jesus parallels Deut 19:15 by referencing how one ought to regard a brother who sins against another and how to resolve that conflict. The wording is such that a well versed Jew would know it to be referencing Deuteronomy.

Jesus quotes Deut 6:5 as the "greatest commandment" in Matthew 22:38. This is an answer to a question which is posed to him by the Pharisees and one of the two commandments Jesus says that "depend all the Law and the Prophets."

\section{Summary}

The Pentateuch is used within Matthew in two primary ways. First, it is often used in the context of the sermon on the mount as "you have heard it said.." references that Jesus then interprets, always in such a way to be more difficult than the commands originally seem. 

Second, it is used as a quotation in defense or debate. Jesus often quotes this when engaging with Pharisees, Sadducees, and scribes. He also quotes it in response to the devil's tempting. When combined with the various interpretations of scripture, it shows the clear authority and knowledge of scripture that Christ possessed. This is amplified by  references to people being in awe of this teaching (7:27-29, 22:33).


\newpage
\printbibliography

\end{document}
