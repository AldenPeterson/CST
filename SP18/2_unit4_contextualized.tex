
\documentclass[12pt]{turabian-researchpaper}
\usepackage{biblatex-chicago}
\renewcommand{\thesection}{\Roman{section}}
\renewcommand{\thesubsection}{\thesection.\Roman{subsection}}

\renewcommand{\bibsetup}{\singlespacing}
\renewcommand{\bibitemsep}{1\baselineskip}
\renewcommand{\bibhang}{0.5in}	

\addbibresource{ntsurvey.bib}


\begin{document}


\begin{singlespace}
\noindent Alden Peterson \newline
\noindent SP18 - NT Survey 2 - Romans to Revelation\newline
\noindent Unit 4 - Contextualized Assignment
\newline\noindent\rule{4cm}{0.4pt}
\end{singlespace}


%Compose a 750-1,000 word essay (3-5 pages) arguing for the literary coherence of Philippians.
%Be sure to cite each chapter of the epistle, and the article “Philippians, Letter to the,” in DPL.

\section{Overview of Philippians}

Philippians 1 starts out, as do most of Paul's letters, with an introduction, containing his formal introduction and the Pauline thanksgiving and prayer. This immediately transitions into a clarification of Paul regarding his imprisonment (1:12-18) and its benefit for the gospel. Chapter 1 concludes with the famous "to live is Christ, to die is gain" section (1:19-30) where Paul affirms the importance of living for Christ on behalf of others, such as the Philippians to whom he has authored the letter and commends them to live a life worthy of Christ (1:27-30).

Chapter 2 starts with clarification on how this living a life worthy of Christ looks, that being the example of Christ. Paul gives a few specific examples (2:2-5). As he continues, he encourages them throughout 2:12-2:18 to live in a manner which shines light in the world.

Following this section, Paul shares his future plans of sending Timothy and current plan of sending Epaphroditus to the church and affirms their value and character (2:19-2:30). Paul gives the Philippians strong encouragement to receive them well.

The transition from Chapter 2 to 3 changes into a more clear warning and command, that being that our righteousness comes from Christ - not our works. Paul makes it obvious how important this is by his descriptions of those that might issue doctrine contrary, calling them "dogs" and "evil doers." This warning (3:2-3:7) is significantly different in tone from the previous part of the letter and is followed by a beautiful and clear explanation of the process of sanctification in 3:8-3:16. 

After this sanctification description, Paul again exhorts the believers to imitate him and provides more warning of the earthly and spiritual consequences of those who do now walk as enemies of Christ (3:17-3:21).

The book concludes with Chapter 4, which is an exhortation for the church (4:1-4:9) and thanksgiving by Paul that the brothers in Philippi were with him in his time of need, via gifts sent through Epaphroditus.

\section{Themes \& Coherence}

As DPL points out there are clear themes within the book of Philippians resulting in considerable theological depth to the book, in spite of the relatively short length of the book.\autocite[pg. 712]{hawthorne2009dictionary}

The first theme I will discuss is my favorite of the book: joy. Reading through Philippians you cannot help but notice Paul's \textit{joy} - he starts out professing thankfulness for his current imprisonment (1:12-1:14), followed by a section on rejoicing over the situation (1:18b-1:26). Paul commends the reader to rejoice again in 3:1 and 4:4 and concludes his letter with a length thanksgiving (4:10-4:20). This theme of joy is all the more remarkable as a reader considering the circumstance Paul found himself in when writing - in prison.

In addition to the multiple references to joy itself there are several passages which are are reflectant of hymn style.\autocite{hawthorne2009dictionary}

Philippians also contains an incredibly clear statement related to sanctification, found in 3:8-3:16\autocite[pg. 713]{hawthorne2009dictionary}, which also builds on context from chapters one and two. The clarity of the statement is built upon and central to many of the exhortations given to the reader as well as to central to why Paul has such joy.

A final key theme within the book is that of exhortation. Paul exhorts the reader to live in light of this words on rejoicing in the Lord in numerous occasions throughout the book. He does his exhortations in a very positive and encouraging manner (contrasted with other epistles such as 1st Corinthians) and is overwhelmingly using this framework for his exhortations. These largely speaking are related to how the believers ought to interact with each other (2:2-2:5, 2:25-2:30, 3:17-3:21, 4:2-4:3). 

\section{Closing Thoughts}
Last, and perhaps most relevant to most readers, the book of Philippians just flows well as a cohesive text. The theme of joy shines through the pages of the book in a way which feels self evident to me. While not a strictly scholarly explanation by itself I think it is worth mention. Some scripture has a strong self evidency regarding Truth or cohesion; I would argue that Philippians is among the most clearly self evident reflections of the grace of God and the joy that we as believers can have among all the New Testament. This is a beautiful theme so clearly and inspirationally written about it makes a reader simply ``amen`` along with Paul when he writes in closing,``To our God and Father be glory for ever and ever.''
\newpage


\printbibliography

\end{document}
