
\documentclass[12pt]{turabian-researchpaper}
\usepackage{biblatex-chicago}
\renewcommand{\thesection}{\Roman{section}}
\renewcommand{\thesubsection}{\thesection.\Roman{subsection}}

\renewcommand{\bibsetup}{\singlespacing}
\renewcommand{\bibitemsep}{1\baselineskip}
\renewcommand{\bibhang}{0.5in}	

\addbibresource{ntsurvey.bib}


\begin{document}


\begin{singlespace}
\noindent Alden Peterson \newline
\noindent SP18 - NT Survey 1 - Gospels and Acts\newline
\noindent Unit 4 - Scholar Track\newline
\noindent\rule{4cm}{0.4pt}
\end{singlespace}



%One of the important skills a teacher must have is the ability to find and summarize the content of research sources for future reference. This activity allows you to practice this while demonstrating your knowledge of the Gospel of Luke.

%Find 3 articles (journals or book chapters) regarding Luke's theological emphases (Christology, Pneumatology, Kingdom of God, etc.)
%Compose an annotated bibliography (at least 10 sentences each) of the articles
%Make sure your bibliographic entries conform to Turabian standards

\section{Dictionary of Jesus and the Gospels\autocite[pg. 540-552]{green2013dictionary}
}

In the DJG text, Green introduces the primary theological emphasis as that of salvation. Many of the sub-themes are related to this, such as the discussion on God and God's purpose. Adding to that is discussion of Jesus as God's son and his status as Messiah and savior. A third theme is the focus on how one ought to respond to God's kingdom. Green also suggests that a significant theme and emphases in Luke is the value placed on women, through the frequency of mention within the chapter. He also expresses that the interaction with Israel is an important emphasis of the text.

While I find these to generally be true, I think that several themes are missing from Green's theological motifs. First, he chooses to focus on several themes based on quantity of reference (value of women, God's kingdom, etc) but does not address either the Holy Spirit or our relationship with money (except as a subpoint within God's Kingdom). This chapter would have benefited inclusion of either of these.

 
 \section{Distinctive Theologies in the Gospel of Luke\autocite{biblicaltraining18}}

The first theme the authors of these lecture notes suggest is a theme for stewardship of material possessions, both with respect to managing your own as well as our need to attend to the poor. The next theme is the interaction of Jews and the law, begun in Luke and continued throughout the book of Acts. This represents a major positive I found from this source - the authors chose to treat Acts/Luke as effectively a two part work, drawing themes from both as a cohesive work rather than investigating the book of Luke in isolation from Acts. Also suggested is the lesser emphasis on an imminent eschatology, particularly when compared with other gospels. The last two theological topics the authors point out are the role of Satan and God working through the Holy Spirit. Satan is heavily referenced in the book, both as demons and Satan himself. The authors point out the Holy Spirit is mentioned considerably more in Luke than in either Matthew or Mark.

As previously mentioned, this source heavily referenced the combined work of both Acts and Luke when determining the theological themes coming from Luke. Initially jarring, I actually found this a pretty interesting perspective to look at the book of Luke. The authors also made comparisons of the content of Luke to Matthew/Mark. While useful, I found this distracting - the theme of Luke should be discernible independent of the other gospels.

\section{Grace to You: Luke\autocite{gty18}}

The first theme that Grace to You discusses is that of Jesus' compassion - to many marginalized people groups such as Samaritans, women, and tax collectors. They make an interesting point that every time a tax collector is mentioned it is positive.  Christ's ministry to the marginalized is a key theological theme. Even outside the longer context of outcasts, the writers specifically mention women, as well, both their role in ministry to the Lord as well as life. Additional themes include human fear of God, forgiveness, joy, wonder at mysteries of divine truth, the role of the Holy Spirit, and the temple in Jerusalem.  Last is an extended theme of Jesus' progression towards the Cross. Luke uniquely writes a large portion of material covering this final journey to Jerusalem with nearly 10 chapters which points to His purpose on Earth.

This summary felt to me to be the best (coincidentally it was the most concise of those I read) summary of the key theological themes within Luke. It also felt appropriately balanced with a mixture of theological as well as descriptive (e.g. Christological emphasis). My only critique would they do not mention anywhere the focus on material possessions, even in passing. I would be curious what they would say if asked about why not even a mention of this was included in an otherwise very solid summary and analysis.

\section{Closing reflection}

Writing this section made me realize that I lack significant numbers of reference materials to cite and incorporate. I suspect my wife and I will be purchasing more books from the recommended list as a result - it was difficult to find concise and citeable sources without finding what ultimately is entire books. As such my sources felt weak for this assignment (not to mention citing the course textbook!).

Additionally, finding specific differences in the presentation of "theological emphasis" vs "themes" was difficult as many sources I found mixed them together nearly inseparably. 

\newpage
\printbibliography

\end{document}
