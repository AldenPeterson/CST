\documentclass[12pt]{turabian-researchpaper}
\usepackage{biblatex-chicago}
\renewcommand{\thesection}{\Roman{section}}
\renewcommand{\thesubsection}{\thesection.\Roman{subsection}}

\renewcommand{\bibsetup}{\singlespacing}
\renewcommand{\bibitemsep}{1\baselineskip}
\renewcommand{\bibhang}{0.5in}	

\addbibresource{ntsurvey.bib}


\begin{document}


\begin{singlespace}
\noindent Alden Peterson \newline
\noindent SP18 - NT Survey 1 - Gospels and Acts\newline
\noindent Unit 1 - Essay  \newline
\noindent\rule{4cm}{0.4pt}
\end{singlespace}


%Argue for the 27 books of the NT as the only valide NT texts


\section{Reflection}
%Reflect on the identification of the NT Canon as described in the articles "Canon" and "Canonical Criticism"
Prior to this course, I had fairly limited (and naive and inaccurate!) understanding of how the overall canonization process had worked. If asked I likely would have said something like, "at the Council of Nicaea in the 4th century people discussed and agreed upon the current New Testament, resulting in what we know today as the New Testament." 

Recognizing now from both the assigned readings as well as the scholarly research how inaccurate and wrong this was is, to be honest, somewhat depressing. Seeing how much canonization happened unofficially and considerably earlier than the Council of Nicaea lends to a much stronger answer to objections regarding the canonization of the New Testament. In particular, the historical and canonical understanding of how the four Gospels came to be, which are arguably the most important of the New Testament texts. Evidence such as the various codices found containing the Biblical gospels and Acts strongly suggest their collective status within the early church as canon.

I did not know how much historical evidence there is in favor of these books being authentic as well as historically adopted as canon. Even things such as learning more about how often the current NT canon was read in public (similar to the method of reading the Old Testament or the Pentateuch) in the early Church lends strong credence to the treatment of these books as canon. Additionally learning how much discussion has been documented from the first few centuries was interesting to me; I was unaware any of this had happened and that writings from writers such as Iranaeus exists from the period in-between the time of writing and the Council of Nicaea (which I had erroneously assumed was where the canon was \textit{decided} rather than affirmed).

A last observation in affirmation of the New Testament canon was clarification regarding a lingering question I had not recognized I had - that being, how did the non-apostles such as Mark or Luke get the content for their gospels? And if so, how can we know it to be accurate? Seeing discussion on this, such as the plausibility of Mark having spent time together with Peter addresses that objection well.

\section{New Testament Canonization}

An important aspect to the New Testament canon is the apostleship of the writers, as pronounced by Jesus himself (via verses such as John 14:26). This means that the original apostles can be considered similar to the prophets of the Old Testament - Paul's epistles for example.

In addition to the this, the existing New Testament canon was frequently and historically assumed to be Scripture. Writers often associated the canon together as well as the texts being grouped together physically in various forms. The early church thus had a significant impact on which books were canonized, based on the books which were used, quoted, and written within that context and by approximately A.D. 175 this canon was fairly established. 


Several early sources, such as Caius, who questioned the inclusion of Hebrews, still affirmed the remainder of the NT as canon.


Authorship is a factor too. As previously discussed, the ability of the apostolic writers to speak as prophets results in their ability to actually speak God's word similar to the Old Testament writers. Much of the New Testament is written in this fashion, by apostles, excluding Mark, Luke/Acts, Hebrews, and Jude. 

It is important to note that of these traits shared by the current canon, none of the non-canon books (even those which are controversial) share the strong and common themes. It is worth noting that while the Catholic Bible contains additions to the Old Testament as compared with Protestant Bibles, their New Testament contains the same 27 books.

\newpage
\printbibliography

\end{document}
