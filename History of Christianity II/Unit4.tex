
\documentclass[12pt]{turabian-researchpaper}
\usepackage{biblatex-chicago}
\renewcommand{\thesection}{\Roman{section}}
\renewcommand{\thesubsection}{\thesection.\Roman{subsection}}

\renewcommand{\bibsetup}{\singlespacing}
\renewcommand{\bibitemsep}{1\baselineskip}
\renewcommand{\bibhang}{0.5in}	

\addbibresource{historyii.bib}


\begin{document}


\begin{singlespace}
\noindent Alden Peterson \newline
\noindent M-HT3101 - History of Christianity II\newline
\noindent Unit 4 - Source Material Assignment
\newline\noindent\rule{4cm}{0.4pt}
\end{singlespace}


%The student should comment on the source of the extract (book, letter or other document), author (if known), the significance of the extract given, the context of the statement (persecution, pagan criticism, etc) and any other details the student believes relevant to share.

%.\autocite[pg.253]{woodbridge2013} 
% \autocite[pg.266]{bettenson2011documents}

\section{Presbyterianism: The Westminster Confession of Faith (Section XI - IV)}

In 1643, a group of Englishmen met in the Westminster Abbey to codify advice regarding theological issues such as worship, doctrine, and church government as would relate to the Church of England.\autocite[pg.277]{woodbridge2013} This assembly was called within context of English/Scottish political conflict and part of a process towards reconciliation between the two powers.\autocite[pg.276]{woodbridge2013} 

The assembly, which lasted six years, resulted in a large number of documents which would guide the Church of England and Church of Scotland and were adopted by both countries.\autocite[pg.278]{woodbridge2013} One of the documents produced in the Westminster Assembly was the Westminster Confession. This document which is closely associated with the Presbyterian denomination in the United States as a historic document.\autocite[pg.306]{bettenson2011documents} It has remained influential in Scotland to this day as the most popular religion in Scotland is a Presbyterian denomination to this day.\autocite{wiki:scotlandReligion}

In the late 1600s, the documents from the Westminster Assembly (including the Westminster Confession of Faith) also influenced the Particular Baptists as in 1677 they adopted parts of the documents from the assembly but differed regarding "ordinances, ecclesiology, and relations between church and state."\autocite[pg.402]{woodbridge2013} 


\section{The Baptist Confession of Faith: Second Confession (Section XI - V - B)}

Following barely 30 years after the Westminster Confession of Faith (1643), Particular Baptists again met and drafted what is known as the Second London Confession in 1677.\autocite[pg.402]{woodbridge2013} This had at its core several disagreements with the Presbyterian theology. Some of these disagreements were:\autocite[pg.402]{woodbridge2013} 

\begin{itemize}

\item Christ died only for the elect and his substitutionary atonement only was for elect
\item Believers baptism (not infant)
\item Congregation governs church, not the Presbyterian approach
\end{itemize}

Shortly after this, in 1689, England passed the Act of Toleration which allowed those who were not Anglicans to gain freedom of worship, as long as they swore allegiance to the crown and renounced teachings of the Roman Catholic Church.\autocite[pg.402]{woodbridge2013}  The freedom allowed by the act allowed the independent churches to grow in number such that in 1720, around 230 churches existed in England and Wales.\autocite[pg.402]{woodbridge2013} 

This theological line exists to the present day in the Reformed Baptist as well as within the Southern Baptist Convention.\autocite{wiki:reformedBaptists} It is interesting to note that even today, there are divisions within the Southern Baptist Convention regarding some of the conversation topics discussed in the Second Confession, although strictly speaking SBC is not directly descendant from the Particular Baptist movement.\autocite{wiki:reformedBaptists}

\newpage
\printbibliography

\end{document}
