
\documentclass[12pt]{turabian-researchpaper}
\usepackage{biblatex-chicago}
\renewcommand{\thesection}{\Roman{section}}
\renewcommand{\thesubsection}{\thesection.\Roman{subsection}}

\renewcommand{\bibsetup}{\singlespacing}
\renewcommand{\bibitemsep}{1\baselineskip}
\renewcommand{\bibhang}{0.5in}	

\addbibresource{historyii.bib}


\begin{document}


\begin{singlespace}
\noindent Alden Peterson \newline
\noindent M-HT3101 - History of Christianity II\newline
\noindent Unit 6 - Source Material Assignment
\newline\noindent\rule{4cm}{0.4pt}
\end{singlespace}


%The student should comment on the source of the extract (book, letter or other document), author (if known), the significance of the extract given, the context of the statement (persecution, pagan criticism, etc) and any other details the student believes relevant to share.

%.\autocite[pg.253]{woodbridge2013} 
% \autocite[pg.266]{bettenson2011documents}

\section{Doctrine of Papal Infallibility (Section X - IX)}

The Doctrine of Papal Infallibility was devised in a council called by Pius IX, known as Vatican I.\autocite[pg.626]{woodbridge2013}  The council was designed to uproot “current errors” and was held shortly after Dollinger published an anonymous rebuttal to the Syllabus of Errors in 1869, released by Pius IX several years prior.\autocite[pg.626]{woodbridge2013}  Pius IX had included statements to the effect of papal infallibility in this document and the output of the council was clear that the previous wording would be upheld.\autocite[pg.626]{woodbridge2013}  One effect of the teaching is that the council would no longer be able to challenge or reform the pope when he enacts positions from his position of responsibility.\autocite[pg.626]{woodbridge2013} 

Vatican I was also held in a time where there was considerable global unrest resulting from modernism and liberalism (as well as many revolutions in Europe). The result was definitive affirmation that the Pope of the Catholic church, having seen the string of succession from Peter himself. The Catholic Church uses John 21:15-17, Luke 10:16/22:32, and Matthew 16:16-19/18:18 as scriptural basis for affirming the doctrine, based on the pope being the supreme shepherd of our faith.\autocite{papalinfallibility}

Within the Council, it is worth noting that not all Catholic delegates agreed with the decision - approximately 20\% of the delegates did not vote on this issue (either because they disagreed or because they did not believe the time correct).\autocite[pg.626]{woodbridge2013}  This is significant in light of the existing turmoil surrounding Christianity and worldviews - even within the Catholic church itself, a significant minority disagreed with this particular resolution of papal infallibility.


\newpage
\printbibliography

\end{document}
