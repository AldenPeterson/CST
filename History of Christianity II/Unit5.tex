
\documentclass[12pt]{turabian-researchpaper}
\usepackage{biblatex-chicago}
\renewcommand{\thesection}{\Roman{section}}
\renewcommand{\thesubsection}{\thesection.\Roman{subsection}}

\renewcommand{\bibsetup}{\singlespacing}
\renewcommand{\bibitemsep}{1\baselineskip}
\renewcommand{\bibhang}{0.5in}	

\addbibresource{historyii.bib}


\begin{document}


\begin{singlespace}
\noindent Alden Peterson \newline
\noindent M-HT3101 - History of Christianity II\newline
\noindent Unit 5 - Source Material Assignment
\newline\noindent\rule{4cm}{0.4pt}
\end{singlespace}


%The student should comment on the source of the extract (book, letter or other document), author (if known), the significance of the extract given, the context of the statement (persecution, pagan criticism, etc) and any other details the student believes relevant to share.

%.\autocite[pg.253]{woodbridge2013} 
% \autocite[pg.266]{bettenson2011documents}

\section{The Syllabus of Errors (Section X - VIII)}

The Syllabus of Errors was published in 1864 by the Catholic Pope Pius IX.\autocite[pg.625]{woodbridge2013}  The full title, ``A Syllabus containing the most important errors of our time, which have been condemned by our Holy Father Pius IX in Allocutions, at Consistories, in Encyclicals, and other Apostolic Letters'' gives clarification for what the purpose was - documenting from the pope a litany of errors existing in his present time.

Many of the errors related to the various worldviews being proliferated throughout the 1800's - pantheism, naturalism, absolute rationalism, socialism, and modern liberalism are specifically called out in the document as having errors.\autocite[pg.275-277]{bettenson2011documents} This was in the context of a large number of philosophical and political changes across Europe, namely modernism and revolutionary movements taking place.\autocite{woodbridge2013} The errors were a shot at Protestantism as well, with Error 18 effectively denouncing that Protestants are part of the same faith as the Catholic faith.\autocite[pg.625]{woodbridge2013}

Woodbridge quotes an anonymous Protestant writer as responding saying the Syllabus represents a ``monumental declaration of war against the entirety of science, against the modern state, against contemporary education."\autocite[pg.625]{woodbridge2013} There is not much to add to this summary other than the strong arguments in favor of the Catholic Church present within the Syllabus's eighty clauses.


% In 1643, a group of Englishmen met in the Westminster Abbey to codify advice regarding theological issues such as worship, doctrine, and church government as would relate to the Church of England.\autocite[pg.277]{woodbridge2013} This assembly was called within context of English/Scottish political conflict and part of a process towards reconciliation between the two powers.\autocite[pg.276]{woodbridge2013} 

% The assembly, which lasted six years, resulted in a large number of documents which would guide the Church of England and Church of Scotland and were adopted by both countries.\autocite[pg.278]{woodbridge2013} One of the documents produced in the Westminster Assembly was the Westminster Confession. This document which is closely associated with the Presbyterian denomination in the United States as a historic document.\autocite[pg.306]{bettenson2011documents} It has remained influential in Scotland to this day as the most popular religion in Scotland is a Presbyterian denomination to this day.\autocite{wiki:scotlandReligion}

% In the late 1600s, the documents from the Westminster Assembly (including the Westminster Confession of Faith) also influenced the Particular Baptists as in 1677 they adopted parts of the documents from the assembly but differed regarding "ordinances, ecclesiology, and relations between church and state."\autocite[pg.402]{woodbridge2013} 


\newpage
\printbibliography

\end{document}
