
\documentclass[12pt]{turabian-researchpaper}
\usepackage{biblatex-chicago}
\renewcommand{\thesection}{\Roman{section}}
\renewcommand{\thesubsection}{\thesection.\Roman{subsection}}

\renewcommand{\bibsetup}{\singlespacing}
\renewcommand{\bibitemsep}{1\baselineskip}
\renewcommand{\bibhang}{0.5in}	

\addbibresource{historyii.bib}


\begin{document}


\begin{singlespace}
\noindent Alden Peterson \newline
\noindent M-HT3101 - History of Christianity II\newline
\noindent Unit 3 - Source Material Assignment
\newline\noindent\rule{4cm}{0.4pt}
\end{singlespace}


%The student should comment on the source of the extract (book, letter or other document), author (if known), the significance of the extract given, the context of the statement (persecution, pagan criticism, etc) and any other details the student believes relevant to share.



\section{The Council of Trent: On Justification}

At the Council of Trent (1545 - 1563), the Catholic church developed decrees regarding the Catholic orthodoxy on subjects such as justification and grace, sacraments, and the Eucharist.\autocite[pg.253]{woodbridge2013} The selected propositions (15 in total) represented statements which were anathematized\autocite[pg.266]{bettenson2011documents}, in other words, considered against the current Catholic doctrine. These selections were part of a larger corpus of 33 canons specifically on justification and amongst a significantly larger set of topics addressed by the council, with 25 topics being addressed\autocite{trent11}.

These were written amongst the religious instability of the 1500s, after the death of Martin Luther. It was called by Pope Paul III in response to the Protestant Reformation and was delayed until after the death of Luther by war.\autocite[pg.209]{woodbridge2013} It had the stated goal of, ``rooting out of heresy and the reform of conduct [of the clergy].''\autocite[pg.210]{woodbridge2013} 

A key outcome from the Council was clarification in a formal sense of doctrines which had previously been unconfirmed, including the topic of justification.\autocite[pg.211]{woodbridge2013} Prior to this point the subject of justification had not been discussed and agreed upon by the Catholic church and given the importance of justification to the Protestant movement (and disagreements with the Catholic church). 

\section{Arminianism}

The selected Five Articles of the Remonstrants was written after Jacob Arminius died in 1609 by his supporters, led by Jan Uytenbogaert and Simon Episcopius.\autocite[pg.256]{woodbridge2013} This codified their beliefs in response to disagreement with fellow faculty of Arminius's in Leiden, Franciscus Gomarus, that had existed between the two faculty. Codification of the beliefs led to the Synod of Dort, which was in response to Arminius's beliefs and which ultimately resulted in casting the Arminian perspective as heresy.\autocite[pg.258]{woodbridge2013} This lead to the death of Van Oldenbarnevelt, a Dutch statesman who had upheld the Remonstrant viewpoint presented in these articles. It is important to note the Synod of Dort was exclusively attended by Calvinistic representatives and Arminian delegates were forced out.

Arminius lived in the Netherlands until his death in 1609, becoming a professor of theology in Leiden in 1603.\autocite[pg.256]{woodbridge2013}  Both of his parents died when he was young, his mother when he was an infant and father when he was a teenager.

While after both Arminius and Oldenbarnevelt died, after Maurice of Orange died in 1625 Arminianism was allowed back in the Netherlands and within 10 years a church based on that belief system was formed.\autocite[pg.258]{woodbridge2013}

It is important to note that the Calvin vs Arninian debate did not end in the 1600s but is still a theological discussion even in modern times.

\newpage
\printbibliography

\end{document}
