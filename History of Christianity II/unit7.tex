
\documentclass[12pt]{turabian-researchpaper}
\usepackage{biblatex-chicago}
\renewcommand{\thesection}{\Roman{section}}
\renewcommand{\thesubsection}{\thesection.\Roman{subsection}}

\renewcommand{\bibsetup}{\singlespacing}
\renewcommand{\bibitemsep}{1\baselineskip}
\renewcommand{\bibhang}{0.5in}	

\addbibresource{historyii.bib}


\begin{document}


\begin{singlespace}
\noindent Alden Peterson \newline
\noindent M-HT3101 - History of Christianity II\newline
\noindent Unit  - Source Material Assignment
\newline\noindent\rule{4cm}{0.4pt}
\end{singlespace}


%The student should comment on the source of the extract (book, letter or other document), author (if known), the significance of the extract given, the context of the statement (persecution, pagan criticism, etc) and any other details the student believes relevant to share.

%.\autocite[pg.253]{woodbridge2013} 
% \autocite[pg.266]{bettenson2011documents}

% \autocite{wiki:Birmingham}
\section{Letter from Birmingham Jail (Section XIV - III - A)}

Written by Martin Luther King April 16th, 1963, the quoted excerpt was written while he was in jail in Birmingham.\autocite[pg.360]{bettenson2011documents} Dr. King was in jail due to his actions within the nonviolent campaign undertaken by blacks in the city as a result of rampant racism and negotiations which resulted in a broken promise from the economic leaders in Birmingham.\autocite[pg.361]{bettenson2011documents}

The context in which this was written was a larger scale civil rights movement of Black Liberation that was occurring in the United States.\autocite[pg.735]{woodbridge2013}  It is worth noting that many of the leaders in this movement disagreed with Dr. King's non-violent approach and preferred a much stronger approach, through the Black Power movement.\autocite[pg.735]{woodbridge2013} 

Specifically, the events in Birmingham was described by King as, "the most segregated city in the country."\autocite{wiki:Birmingham} The mostly peaceful protests that transpired over 1963 resulted in raising public awareness for the cause of civil rights as well as accomplishing many of the goals in Birmingham that it set out to do.\autocite{wiki:Birmingham} It is worth noting the role of Bull Connor, who was the police/fire chief in Birmingham, who strongly reacted to the protests and effectively raised support for the civil rights movement as a whole by his reactions to the events in Birmingham.\autocite{wiki:Birmingham}

The letter quoted here was an answer to objections of churchmen who had raised concern regarding the non-violent opposition.\autocite[pg.360]{bettenson2011documents}  It is interesting to note how holistically the church involvement was within the movement, too; many of the places protests were organized and initiated from were actually churches.\autocite{wiki:Birmingham}

\section{A Theology of Liberation (Section XIV - IV - C)}

In the 1960s, alongside the civil rights movement, was a movement known as liberation theology. During this time, the Catholic Pope Paul VI published \textit{Populorum Progressio} in 1967 as an affirmation "that the economics of the world should serve to benefit all humankind and not just the wealthy."\autocite[pg.775]{woodbridge2013}

The document strongly pointed out inequities in how markets worked, resulting in economic disparity:\autocite[pg.364]{bettenson2011documents}

\begin{quote}
As a result, nations where industrialization is limited are faced with serious difficulties when they have to rely on their exports to balance their economy and to carry out their plans for development. The poor nations remain ever poorer while the rich ones become still richer.
\end{quote}

This document was expanded upon at length in the referenced excerpt by the Peruvian priest Gustavo Guti\`{e}rrez, who coined the phrase  "liberation theology" in 1971.\autocite[pg.733]{woodbridge2013} He calls out the economic factors (similar to Pope Paul VI) and writes that development in Latin Americawill happen "only if there is liberation from the domination exercised by the great capitalist countries, and especially by the most powerful, the United States of America."\autocite[pg.368]{bettenson2011documents} 

At the time when Guti\`{e}rrez wrote the document, much of Latin America was undergoing significant economic and in many cases military hardship. His theology was a plea to the Catholic Church to return to the mission of caring for poor and criticizes the perceived pandering to the rich/wealthy.\autocite[pg.734]{woodbridge2013} 

The events transpiring in Latin America were corresponding to a different type of liberation theology in the United States at the time - while poverty and political injustice dominated the Latin Americas, North America was undergoing similar efforts to liberate people from the oppression of racism and sexism.\autocite[pg.733]{woodbridge2013} 
\newpage
\printbibliography

\end{document}
