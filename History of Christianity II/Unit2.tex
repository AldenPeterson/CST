
\documentclass[12pt]{turabian-researchpaper}
\usepackage{biblatex-chicago}
\renewcommand{\thesection}{\Roman{section}}
\renewcommand{\thesubsection}{\thesection.\Roman{subsection}}

\renewcommand{\bibsetup}{\singlespacing}
\renewcommand{\bibitemsep}{1\baselineskip}
\renewcommand{\bibhang}{0.5in}	

\addbibresource{historyii.bib}


\begin{document}


\begin{singlespace}
\noindent Alden Peterson \newline
\noindent M-HT3101 - History of Christianity II\newline
\noindent Unit 2 - Source Material Assignment
\newline\noindent\rule{4cm}{0.4pt}
\end{singlespace}


%The student should comment on the source of the extract (book, letter or other document), author (if known), the significance of the extract given, the context of the statement (persecution, pagan criticism, etc) and any other details the student believes relevant to share.






\section{Calvinism (Part 2 - Section VIII - 2)}
%1) Calvinism (Part 2 - Section VIII - 2)
This excerpt is from Christianae Religionis Institutio (Institutes of the Christian Religion) which was the crowning achievement of John Calvin's life and widely regarded as "one of the greatest theological writings in the history of the Western Church."\autocite[pg.172]{woodbridge2013} John Calvin was leaving Paris en route to Strasbourg when he arrived in Geneva, which at that time was undergoing a significant Protestant reformation (at the end of 1535 all Catholic clergy were forced out or to convert to Protestantism) and where he wrote the first edition of his Institutes.\autocite[pg.158]{woodbridge2013} He continued writing this until 1559 (which is the edition the excerpt is quoted from) where it had eighty chapters.\autocite[pg.173]{woodbridge2013}

Woodbridge describes Calvin as having a significant influence from Luther and expanding on his ideas.\autocite[pg.173]{woodbridge2013} Calvin is most well known for his position on predestination but this was not the center of his theological beliefs.\autocite[pg.173]{woodbridge2013} 

\section{The Supremacy Act (Part 2 - Section IX - I - D)}
%2) The Supremacy Act (Part 2 - Section IX - I - D)

The Act of Supremacy were passed in November of 1534 in England declaring the king to be the "supreme head of the Church of England"\autocite[pg.242]{bettenson2011documents} which effectively broke off all ties organizationally with Rome.\autocite[pg.224]{woodbridge2013} This was shortly after the King had his marriage to Catherine of Aragon annulled by Archbishop Thomas Cranmer,
an act that bypassed Rome subsequent to the Act in Restraint in Appeals.\autocite[pg.224]{woodbridge2013}

The Supremacy Act is significant because it was the final nail in the coffin of separation between the Church of England and the Catholic Church in Rome (and papal authority). While not expressly intended to introduce Protestantism into England, by separating from Rome England made themselves welcoming towards other who similarly wanted to separate from Rome - and in the 1530s this included many Protestants.\autocite[pg.225]{woodbridge2013}

A lasting significance which has carried forward to our modern day is that several years after this separation, Cromwell and Cranmer received permission from King Henry VIII to produce a copy of the Bible in local vernacular in 1537.\autocite[pg.226]{woodbridge2013} The break with Rome has also persisted to modern times, with the Church of England still remaining separate from the Catholic Church even now.
\newpage
\printbibliography

\end{document}
